\section{Noncommutative probability}


\subsection{From commutative to noncommutative}

Consider the Hilbert space $\calH=L_\bbC^2(\Omega,\mu)$ over the complete probability space $(\Omega,\calF,\mu)$. Then any function $f\in L_\bbC^\infty(\Omega,\mu)$ gives rise to a multiplication operator $\sfM_f\in \calB(\calH)$:
\[
	\sfM_f \,g = fg \in \calH\qquad\forall\,g\in\calH,
\]
with $\|\sfM_f\|_\infty = \|f\|_{L^\infty(\mu)}$. The collection of all such multiplication operators
\[
	\scrA := \bigl\{ \sfM_f:f\in L_\bbC^\infty(\Omega,\mu) \bigr\}\subset\calB(\calH)
\]
forms a \emph{commutative} subalgebra of $\calB(\calH)$. 

\medskip
In fact, this subalgebra is a possibly noncommutative \emph{von Neumann} algebra:

\begin{definition}[von Neumann algebra]
	A (unital) \emph{von Neumann algebra} is a $*$-subalgebra $\scrA\subset \calB(\calH)$ that contains $I_\calH$ and is closed in the \emph{weak operator topology} (\textsc{wot}), i.e.,
	\[
		\textsc{wot-\!}\lim \oba_n = \oba\quad\Longleftrightarrow\quad \langle f,\oba_ng\rangle_\calH \to \langle f,\oba g\rangle_\calH\quad\forall f,g\in\calH.
	\]
	Equivalently, is it a $C^*$-algebra with a predual $\scrA_*$ that is a Banach space.
\end{definition}

There are sub-families of $\scrA$ that play a distinguished role, namely,
\begin{align*}
	\scrA_+ &:= \bigl\{ \oba\in \scrA : \oba\succeq 0 \bigr\} &\text{(nonnegative operators)}\\
	\scrO &:= \bigl\{ \oba\in \scrA : \oba^\dagger = \oba \bigr\} &\text{(Hermitian operators)} \\
	\scrP &:= \bigl\{ \oba\in \calO : \oba^2 = \oba \bigr\} &\text{(projection operators)}
\end{align*}
Here, $\oba\succeq 0$ if and only if $\langle f,\oba f\rangle_\calH\ge 0$ for all $f\in\calH$. Additionally, for von Neumann algebras, we also have the following useful density result:

\begin{proposition}
	Let $\scrA$ be a von Neumann algebra. Then 
	\[
		\scrA = \overline{\text{span}\,(\scrP)}^{\textsc{wot}},
	\]
	i.e., $\scrA$ is the \textsc{wot}-closure of the linear span of its projections.
\end{proposition}

Let us return to our previous example with $\scrA=\{ \sfM_f:f\in L_\bbC^\infty(\Omega,\mu)\}$. It is not difficult to see that $\scrA$ is a $W^*$-algebra. Notice that since $\sfM_f$ is self-adjoint for $f\in L_\bbR^\infty(\Omega,\mu)$, the family of self-adjoint operators is given by $\scrO=\{\sfM_f: f\in L_\bbR^\infty(\Omega,\mu)\}$ and the family of projection operators is given by $\scrP = \{\sfM_f\in\scrO : f=\mathbf{1}_A,\,A\in\calF\}$. 

\medskip
On a von Neumann algebra $\scrA$, we can talk about special types of continuous linear functionals on $\scrA$, called states.

\begin{definition}
	A \emph{state} on a von Neumann algebra $\scrA$ is a linear functional $\uppsi:\scrA\to \bbC$ that is positive and normalized, i.e., $\uppsi(\oba^\dagger \oba)\ge 0$ for all $\oba\in\scrA$ and $\uppsi(I_{\calB(\calH)})=1$.
	
	A state $\uppsi$ is said to be
	\begin{itemize}[itemsep=0.1em]
		\item[] \emph{faithful} if $\uppsi(\oba^\dagger \oba)=0\;\Leftrightarrow\; \oba=0$,
		\item[] \emph{tracial} if $\uppsi(\oba \obb)=\uppsi(\obb \oba)$ for all $\oba,\obb\in\scrA$, and
		\item[] \emph{normal} if $\uppsi\in\scrA_*$, i.e., if it is an element of the predual $\scrA_*$.
	\end{itemize}
\end{definition}



For any probability measure $\nu \ll \mu$, we set
\[
	\uppsi_\nu(\sfM_f) := \int_\Omega f \dd\nu,\qquad f\in L_\bbC^\infty(\Omega,\mu),
\]
we find that $\uppsi$ is a linear functional that is positive and normalized, i.e., $\uppsi$ is a state. Moreover, it is tracial. It is normal if $\omega:=d\nu/d\mu \in L^1(\Omega,\mu)$ and faithful if $\omega>0$.

\medskip
Normal states play an essential role, serving as a counterpart to classical measures, as made explicit by the following proposition.

\begin{proposition}\label{prop:normal-state}
	Let $\uppsi$ be a state on a von Neumann algebra $\scrA$. The following are equivalent:
	\begin{enumerate}[label=(\roman*),itemsep=0.1em]
		\item $\uppsi$ is a normal state.
		\item ($\sigma$-additivity) If $(\oba_n)\subset\calP$ are mutually orthogonal projections, i.e., $\oba_n(\calH)\perp \oba_m(\calH)$ for all $n\ne m$, and $\oba=\vee_n\, \oba_n$ being the projection on the smallest closed subspace containing $\cup_n\, \oba_n(\calH)$, then
		\[
			\uppsi(\oba) = \sum\nolimits_n \uppsi(\oba_n).
		\]
		\item (Continuity from below) For any increasing net $0\preceq \oba_n \uparrow \oba$ in $\scrA_+$, one has the increasing limit $\uppsi(\oba_n)\uparrow\uppsi(\oba)$.
		\item There exists a family $\{\xi_n\}\subset\calH$ with $\sum_n \|\xi_n\|_\calH^2 = 1$ such that\footnote{[Theorem 7.1.8, Fundamentals of the Theory of Operator Algebras, V.II, Kadison-Ringrose]}
		\[
			\uppsi = \sum\nolimits_n \langle \xi_n,\cdot\, \xi_n\rangle\qquad\text{in the sense of norm convergence}.
		\]
	\end{enumerate}
\end{proposition}

%\begin{definition}
%		A von Neumann algebra $\scrA$ is said to be \emph{$\sigma$-finite} if it admits at most countably many orthogonal projections.
%\end{definition}
%
%\begin{proposition}
%	Let $\scrA$ be a von Neumann algebra. Then the following are equivalent:
%	\begin{enumerate}[label=(\roman*),itemsep=0.1em]
%		\item $\scrA$ is $\sigma$-finite.
%		\item There exists a countable separating subset of $\calH$ for $\scrA$.
%		\item There exists a faithful positive linear functional in $\scrA_*$.
%	\end{enumerate}
%\end{proposition}


Consider an arbitrary normal state $\uppsi$ and set
\[
	\mu(A) := \uppsi(\sfM_{\mathbf{1}_A})\qquad\text{for every $A\in\calF$}.
\]
Then, clearly, $\mu(\emptyset)=0$, $\mu(\Omega)=\uppsi(I_{\calB(\calH)})=1$ and $\mu$ is $\sigma$-additive due to the equivalent characterization of a normal state given by Proposition~\ref{prop:normal-state}(ii). In particular, one obtains a classical measure on $(\Omega,\calF)$. In this sense, a state on a noncommutative von Neumann algebra generalizes that of a classical measure.

%\medskip
%Along this line of thought, one could define a probability distribution on $\scrP$ analogously. Indeed, for a normal state $\uppsi$ on a von Neumann algebra $\scrA$, one sets
%\[
%	\mu(a)=\uppsi(a)\qquad\forall\,a\in\scrP.
%\]
%If $\chi_\mu(\psi):=\mu(|\psi\rangle\langle \psi|)$ where $\|\psi\|_\calH=1$, then $\chi_\mu(c\psi)=\chi_\mu(\psi)$ for any $c\in\bbC$ with $|c|=1$. Moreover, for any orthonormal basis $\{\psi_j\}\subset\calH$, we have that $\sum_j |\psi_j\rangle\langle \psi_j| = I_\calH$, and thus $\sum_j \chi_\mu(\psi_j)=1$. Gleason's theorem\footnote{Parthasarathy} states that for Hilbert spaces with dimension $d\ge 3$, there exists a bounded operator $T_\mu$ on $\calH$ such that $\chi_\mu(\psi)=\langle \psi,T_\mu\psi\rangle$ for all $\|\psi\|=1$. 
%
%\begin{proposition}
%	Let $\mu$ be a probability distribution on $\scrP$, where $\dim\calH\ge 3$. Then there exists an orthonormal set $\{\psi_j\}\subset\calH$ and positive scalars $\{p_j\}$ with $\sum_j p_j=1$ such that
%	\[
%		\mu(a) = \sum\nolimits_j p_j \langle\psi_j,a\psi_j\rangle\qquad\forall\,a\in\scrP
%	\]
%\end{proposition}

\subsection{Observables}

In classical probability, we are often interested in computing expressions like $\bbP(X\in A)$, where $X$ is a random variable and $A\subset\bbR$ is a Borel set. 

In the quantum world, a random variable is modelled by a Hermitian operator $\oba\in\scrO$ and is called an \emph{observable}. Observables have their spectrum in $\bbR$ and can therefore be \emph{measured}.

Given a $\uppsi$ on a von Neumann algebra $\scrA$, the expectation of an observable $\oba$ w.r.t.\ the state $\uppsi$ is given by $\uppsi(\oba)$. Since every observable $\oba\in\scrO$ has a spectral decomposition
\[
	\oba = \int_\bbR \lambda\, E_\oba(d\lambda),\qquad E_\oba \,\widehat{=}\, \text{$\scrP$-valued measure,}
\]
we can associate a classical probability measure with the observable $\oba$ and state $\uppsi$:
\[
	\bbP_\uppsi(\oba\in A) := \uppsi(E_\oba(A))\qquad\text{for all Borel set $A\subset\bbR$}.
\]
For two observables $\oba,\obb\in\scrO$ that do not commute, one would like to be able to write $\bbP_\uppsi(\oba\in A,\obb\in B)$ for two Borel sets $A,B\subset\bbR$. The simple argument for this is that if $[\oba,\obb]\ne 0$, then $E_\oba(A)$ and $E_\obb(B)$ may not commute for all Borel sets $A, B$. In particular, 
\[
	\uppsi(E_\oba(A)E_\obb(B)) \ne \uppsi(E_\obb(B)E_\oba(A)),
\]
so there is no consistent way of writing $\bbP_\uppsi(\oba\in A,\obb\in B)$. Yet, when they do commute, then $\uppsi(E_\oba(A)E_\obb(B))=\uppsi(E_\obb(B)E_\oba(A))$ and a joint distribution exists and we are back to the classical scenario. 

\medskip
However, this turns out not to be possible, as the following example portrays.


\begin{example}[Stern-Gerlach measurements] 

Take a beam of atoms (each with spin-$\tfrac{1}{2}$) and perform the following steps:
\begin{figure}[h]\centering
\begin{tikzpicture}[
    >=Latex,
    beam/.style={line width=1.0pt},
    magnet/.style={draw,rounded corners,minimum width=14mm,minimum height=22mm,thick,align=center,fill=gray!10},
    block/.style={draw,thick,minimum width=10mm,minimum height=6mm,fill=red!10,align=center},
%    det/.style={draw,thick,circle,minimum size=3mm,fill=black!70},
    note/.style={font=\footnotesize}
]

% --- Positions ---
% Source and first SG_z
\coordinate (S) at (-2,0);
\node[magnet] (SGz1) at (0,0) {SG$_z$};

% After SGz1 split
\coordinate (zup)   at (2, 0.9);
\coordinate (zdown) at (2,-0.9);

% Block on the down arm
\node[note] (Blk) at ($(zdown)+(1,0)$) {discard arm};

% SG_x on the up arm
\node[magnet,right=1.5cm of zup] (SGx) {SG$_x$};

% After SGx split
\coordinate (xup)   at ($(SGx.east)+(1.5, 0.9)$);
\coordinate (xdown) at ($(SGx.east)+(1.5,-0.9)$);

% Choose the +x branch into final SG_z
\node[magnet,right=2.0cm of xup] (SGz2) {SG$_z$};

% After SGz2 split
\coordinate (zup2)   at ($(SGz2.east)+(2.0, 0.9)$);
\coordinate (zdown2) at ($(SGz2.east)+(2.0,-0.9)$);

% --- Beams ---

% Source to first SG_z
\draw[beam,->] (S) -- (SGz1.west) node[pos=0.9,above left=1pt and 1pt,note] {source};

% SG_z split
\draw[beam,->] (SGz1.east) -- (zup)   node[midway,above,note] {$\ket{\uparrow_z}$};
\draw[beam,->] (SGz1.east) -- (zdown) node[midway,below,note] {$\ket{\downarrow_z}$};

% Block the down arm
%\draw[beam] (zdown) -- (Blk.west);
%\draw[very thick] ($(Blk.north west)+(0.5mm,0)$) -- ($(Blk.south east)+(-0.5mm,0)$);
%\draw[very thick] ($(Blk.south west)+(0.5mm,0)$) -- ($(Blk.north east)+(-0.5mm,0)$);

% Up arm into SG_x
\draw[beam,->] (zup) -- (SGx.west) node[midway,above,note] {filter $\ket{\uparrow_z}$};

% SG_x split (50/50)
\draw[beam,->] (SGx.east) -- (xup)   node[midway,above,note] {$\ket{\uparrow_x}$ };
\draw[beam,->] (SGx.east) -- (xdown) node[midway,below,note] {$\ket{\downarrow_x}$};

% Route +x arm to final SG_z
\draw[beam,->] (xup) -- (SGz2.west) node[midway,above,note] {select $\ket{\uparrow_x}$};

% Final SG_z split (50/50 again)
\draw[beam,->] (SGz2.east) -- (zup2)   node[midway,above,note] {$\ket{\uparrow_z}$};
\draw[beam,->] (SGz2.east) -- (zdown2) node[midway,below,note] {$\ket{\downarrow_z}$};

% Detectors (optional)
%\node[det] at (zup2) {};
%\node[det] at (zdown2) {};
%\node[det] at (xdown) {};

% Labels on devices
%\node[note,below=2mm of SGz1] {measure $z$};
%\node[note,below=2mm of SGx]  {measure $x$};
%\node[note,below=2mm of SGz2] {measure $z$};

% Explanatory notes
%\node[note,below right=1mm and -2mm of zdown] {discard arm};
\node[note] at ($(xdown)+(1,0)$) {unused arm};
\end{tikzpicture}
\caption{Stern-Gerlach experiment}
\end{figure}


\begin{enumerate}
	\item[\textbf{SG$_z$}] Measure spin in $z$-direction: Send the beam through a Stern-Gerlach magnet oriented along $z$. The beam splits into spin-up $\ket{\uparrow}$ and spin-down $\ket{\uparrow}$ paths. Keep only the spin-up branch, which gives a pure state $\ket{\uparrow_z}$.

	\item[\textbf{SG$_x$}] Measure spin in $x$-direction: Now send the filtered beam through a second magnet, oriented along $x$. The beam splits again into $\ket{\uparrow_x}$ and $\ket{\downarrow_x}$ outcomes with $50\%$--$50\%$ probability. So far, this can \emph{still} be explained with classical probability.
	
	\item[\textbf{SG$_z$}] Measure spin in $z$-direction again: Send either of the filtered beams $\ket{\uparrow_x}$ or $\ket{\downarrow_x}$ through a $z$-magnet again. You do \emph{not} get back the original result, i.e., instead of remaining spin-up, you see another $50\%$--$50\%$ split between $\ket{\uparrow_z}$ and $\ket{\downarrow_z}$
\end{enumerate}

Morally, if spin-$x$ and spin-$z$ were classical random variables, measuring $x$ would \emph{erase} knowledge of $z$. In classical probability, observing one property never randomizes another unless there is \emph{hidden causal disturbance}.

Quantum mechanically, the two observables (or random variables)
\begin{align*}
	\sigma_z = \begin{bmatrix}
		1 & 0 \\ 0 &-1
	\end{bmatrix}\quad\text{spin in $z$-direction},\qquad 
	\sigma_x = \begin{bmatrix}
		0 & 1 \\ 1 &0
	\end{bmatrix}\quad\text{spin in $x$-direction}.
\end{align*}
do not commute, i.e., $[\sigma_x,\sigma_y]\ne 0$, and measuring $\sigma_x$ \emph{changes} information about $\sigma_z$. In this sense, they cannot be jointly sampled, i.e., there is no classical joint probability $p(\sigma_x,\sigma_z)$ for these observables, which illustrates the need for noncommutative probability.
\end{example}






