\section{Stochastic Dilation}



\subsection{Noncommutative random variables}

\begin{definition}\label{def:random-variable}
	A \emph{noncommutative $\frX$-valued random variable} on a quantum probability space $(\frA,\mu)$ is an identity preserving $*$-homomorphism
	\[
		\frz\colon \frX\to\frA.
	\]
	Correspondingly, a \emph{noncommutative $\frX$-valued stochastic process} on a quantum probability space $(\frA,\mu)$ is a family of random variables
	\[
		\frz_t\colon \frX\to \frA,\quad t\in\bbT,
	\]
	with $\bbT$ being a (possibly uncountable) index set.
\end{definition}

\begin{example}[Classical random variable]
	Let $X$ be a classical $E$-valued random variable on $(\Omega,\bbP)$. Setting
\[
	\frX\coloneqq B_b(E),\qquad \frA\coloneqq L^\infty(\Omega,\bbP),\qquad \mu(g)\coloneqq \int_\Omega g\dd\bbP,\quad g\in\frA,
\]
we have that
\[
		\frX\ni f\mapsto \frz(f) \coloneqq f\circ X \in L^\infty(\Omega,\bbP)
\]
is a noncommutative $\frX$-valued random variable. We see here that the noncommutative notion of a random variable is `dual' to the classical notion of a random variable.
\end{example}

\begin{example}[Classical stochastic process]\label{ex:stochastic-classical} Consider the space of continuous paths (or trajectories) $\Omega\coloneqq \calC_0(\bbR_+;E)$ starting at $0$ and the Wiener measure $\sfR$. Let $X=(X_t)_{t\in \bbT}$ be the canonical stochastic process $X_t(\omega)=\omega(t)$, $t\in\bbR_+$. As before, we  set
\[
	\frX \coloneqq B_b(E),\qquad \frA\coloneqq L^\infty(\Omega,\sfR),\qquad \mu(G)\coloneqq \int_\Omega G\dd\sfR,\quad G\in\frA.
\]
Then,
\[
	\frX\ni f\mapsto \frz_t(f)\coloneqq f\circ X_t\in L^\infty(\Omega,\sfR)
\]
defines a noncommutative $\frX$-valued stochastic process on $(\frA,\mu)$.
\end{example}

\begin{definition}
	A stochastic process $\frz_t\colon\frX\to (\frA,\mu)$, $t\in\bbT$ admits a \emph{time translation} if there are $*$-homomorphisms $\upalpha_t\colon \frA\to\frA$, $t\in\bbT$ such that
	\begin{enumerate}[label=(\arabic*),itemsep=0em]
		\item $\upalpha_{t+s} = \upalpha_t\circ\upalpha_s$ for all $s,t\in\bbT$, and
		\item $\frz_t=\upalpha_t\circ \frz_0$ for all $t\in\bbT$.
	\end{enumerate}
\end{definition}

\begin{example}
	Consider the Wiener space $(\Omega,\sfR)$ in Example~\ref{ex:stochastic-classical}. Then, $\sfR$ is stationary under the time-shift operator $\frs_t\omega = \omega(\cdot - t)$, i.e., $(\frs_t)_\#\sfR=\sfR$ for every $t\in\bbT$. Hence, the $*$-homomorphism
	\[
		\upalpha_t(G) = G\circ \frs_t,\quad G\in\frA,\;t\in\bbT
	\]
	is a time translation for the corresponding process $(\frz_t)_{t\in\bbT}$ and leaves the state $\mu$ invariant. Indeed, property (1) holds trivially, while property (2) follows from
	\[
		\upalpha_t\circ\frz_0(f) = f\circ X_0\circ\frs_t = f\circ X_t = \frz_t(f),\qquad f\in\frX.
	\]
	Finally,  we conclude with
	\begin{align*}
		\mu(\alpha_t(G)) = \int_\Omega \alpha_t(G)\dd\sfR =  \int_\Omega G\circ\frs_t\dd\sfR = \int_\Omega G\dd(\frs_t)_\# \sfR = \int_\Omega G\dd\sfR = \mu(G),
	\end{align*}
	thus proving the invariance of $\mu$.
\end{example}

For $h\in L^2(\bbT)$ we consider the exponential martingale
\[
	\sfZ_t = \exp\left( -\int_0^t h_r\dd B_r -\frac{1}{2}\int_0^t |h_r|^2\dd r\right),\quad t\in\bbT.
\]
It is easy to see that $(\sfZ_t)_{t\in\bbT}$ satisfies
\[
	\d \sfZ_t = - h_t\sfZ_t\,\d B_t,\qquad \sfZ_0=1.
\]
Setting $\sfQ\coloneqq\sfZ_T\sfP$, we claim that
\[
	W_t = B_t  + \int_0^t h_r\dd r\quad\text{is a $\sfQ$-martingale}.
\]
To see this, we note that $W$ is a $\sfQ$-martingale if and only if $\sfZ_T W$ is a $\sfP$-martingale. Since
\begin{align*}
	\dd(Z_tW_t) &= \d Z_t W_t + Z_t\d W_t + \d [Z,W]_t \\
	&= -h_tZ_t \d B_t + Z_t\d B_t =-(h_t-1)Z_t\d B_t,
\end{align*}
we conclude that $\sfZ_TW$ is indeed a $\sfP$-martingale.


\[
	\d U_t = -\bigl(a^* a\,\d t + i\sqrt{2}\,a\,\d B_t\bigr)U_t
\]


In the classical case, the notion of adaptedness plays an important role for stochastic processes. In this regard, we choose to simply consider the natural filtration induced by a stochastic process. 

\begin{definition}
	Let $\frz_t\colon\frX\to (\frA,\mu)$, $t\in\bbT$ be a stochastic process. For $\bbI\subset\bbT$, we denote by $\frA_\bbI$ the subalgebra of $\frA$ generated by $\{\frz_t(x):x\in\frX,\;t\in \bbI\}$. The filtration induced by $(\frz_t)_{t\in\bbT}$ is then given by $\frF= (\frA_{[0,t]})_{t\in\bbT}$.
\end{definition}

\newpage

\subsection{From commutative to noncommutative}

Consider the Hilbert space $\calH=L_\bbC^2(\Omega,\mu)$ over the complete probability space $(\Omega,\calF,\mu)$. Then any function $f\in L_\bbC^\infty(\Omega,\mu)$ gives rise to a multiplication operator $\sfM_f\in \calB(\calH)$:
\[
	\sfM_f \,g = fg \in \calH\qquad\forall\,g\in\calH,
\]
with $\|\sfM_f\|_\infty = \|f\|_{L^\infty(\mu)}$. The collection of all such multiplication operators
\[
	\calA := \bigl\{ \sfM_f:f\in L_\bbC^\infty(\Omega,\mu) \bigr\}\subset\calB(\calH)
\]
forms a commutative subalgebra of $\calB(\calH)$. 

\medskip
In fact, this subalgebra is a \emph{von Neumann} algebra:

\begin{definition}[von Neumann algebra]
	A (unital) \emph{von Neumann algebra} (or $W^*$-algebra) is a $*$-subalgebra $\calA\subset \calB(\calH)$ that contains $I_\calH$ and is closed in the \emph{weak operator topology} (\textsc{wot}), i.e.,
	\[
		\textsc{wot-\!}\lim a_n = a\quad\Longleftrightarrow\quad \langle f,a_ng\rangle_\calH \to \langle f,ag\rangle_\calH\quad\forall f,g\in\calH.
	\]
	Equivalently, is it a $C^*$-algebra with a predual $\calA_*$ that is a Banach space.
\end{definition}

There are sub-families of $\calA$ that play a distinguished role, namely,
\begin{align*}
	\scrA_+ &:= \bigl\{ a\in \calA : a\succeq 0 \bigr\} &\text{(nonnegative operators)}\\
	\calO &:= \bigl\{ a\in \calA : a^* = a \bigr\} &\text{(self-adjoint operators)} \\
	\calP &:= \bigl\{ a\in \calO : a^2 = a \bigr\} &\text{(projection operators)}
\end{align*}
Here, $a\succeq 0$ if and only if $\langle f,af\rangle_\calH\ge 0$ for all $f\in\calH$.



\medskip
On a von Neumann algebra $\calA$, we can talk about special types of continuous linear functionals on $\calA$, called states.

\begin{definition}
	A \emph{state} on a von Neumann algebra $\calA$ is a linear functional $\uppsi:\calA\to \bbC$ that is positive and normalized, i.e., $\uppsi(a^* a)\ge 0$ for all $a\in\calA$ and $\uppsi(I_{\calB(\calH)})=1$.
	
	A state $\uppsi$ is said to be
	\begin{itemize}[itemsep=0.1em]
		\item[] \emph{faithful} if $\uppsi(a^* a)=0\;\Leftrightarrow\; a=0$,
		\item[] \emph{tracial} if $\uppsi(ab)=\uppsi(ba)$ for all $a,b\in\calA$, and
		\item[] \emph{normal} if $\uppsi\in\calA_*$, i.e., if it is an element of the predual $\calA_*$.
	\end{itemize}
\end{definition}

Let us return to our previous example with $\calA=\{ \sfM_f:f\in L_\bbC^\infty(\Omega,\mu)\}$. It is not difficult to see that $\calA$ is a $W^*$-algebra. Notice that since $\sfM_f$ is self-adjoint for $f\in L_\bbR^\infty(\Omega,\mu)$, the family of self-adjoint operators is given by $\calO=\{\sfM_f: f\in L_\bbR^\infty(\Omega,\mu)\}$ and the family of projection operators is given by $\calP = \{\sfM_f: f=\mathbf{1}_A,\,A\in\calF\}$. Additionally,  

For any probability measure $\nu \ll \mu$, we set
\[
	\uppsi_\nu(\sfM_f) := \int_\Omega f \dd\nu,\qquad f\in L_\bbC^\infty(\Omega,\mu),
\]
we find that $\uppsi$ is a linear functional that is positive and normalized, i.e., $\uppsi$ is a state. Moreover, it is tracial. It is normal if $\omega:=d\nu/d\mu \in L^1(\Omega,\mu)$ and faithful if $\omega>0$.

\medskip
Normal states play an essential role, serving as a counterpart to classical measures, as made explicit by the following proposition.

\begin{proposition}
	Let $\uppsi:\calA\to\bbC$ be a state on a von Neumann algebra $\calA$. The following are equivalent:
	\begin{enumerate}[label=(\roman*),itemsep=0.1em]
		\item $\uppsi$ is a normal state.
		\item ($\sigma$-additivity) If $(a_n)\subset\calP$ are mutually orthogonal projections, i.e., $a_n(\calH)\perp a_m(\calH)$ for all $n\ne m$, and $a=\vee_n\, a_n$ being the projection on the smallest closed subspace containing $\cup_n\, a_n(\calH)$, then
		\[
			\uppsi(a) = \sum\nolimits_n \uppsi(a_n).
		\]
		\item (Continuity from above) For any increasing net $0\preceq a_n \uparrow a$ in $\calA_+$\;$\Rightarrow\;\uppsi(a_n)\uparrow\uppsi(a)$.
	\end{enumerate}
\end{proposition}

Consider an arbitrary normal state $\uppsi$ and set
\[
	\mu(A) := \uppsi(\sfM_{\mathbf{1}_A})\qquad\text{for every $A\in\calF$}.
\]











In particular, we recover the probability $\bbP$ by
\[
	\bbP(A) = \uppsi(\sfM_{\mathbf{1}_A})
\]
\[
	\uppsi(\sfM_{\mathbf{1}_A}\sfM_{\mathbf{1}_B}) = \uppsi(\sfM_{\mathbf{1}_{A\cap B}}) = \bbP(A\cap B)
\]
\[
	\bbP\bigl(\cup_i A_i\bigr) = \uppsi\bigl(\sfM_{\mathbf{1}_{\cup_i A_i}}\bigr) = \uppsi\bigl(\sum\nolimits_i \sfM_{\mathbf{1}_{A_i}}\bigr) = \sum_i \uppsi(\sfM_{\mathbf{1}_{A_i}}) = \sum_i \bbP(A_i)
\]

 

\newpage


Let $(X_t)_{t\in\bbT}$ be a stochastic process on the Wiener space $(\Omega,\sfR)$


Consider a path measure $\sfP$ on $ \Omega\coloneqq\calD(\bbR_+;E)$ and the canonical process $(X_t)_{t\ge 0}$ given by $X_t(\omega)=\omega(t)$, $\omega\in \Omega$, $t\in \bbR_+$. Suppose that for every $f\in B_b(E)$,
\begin{align}\label{eq:classical-martingale}
	f(X_t) - f(X_0) - \int_0^t Lf(X_{r^-})\,\d r \qquad\text{is an $(\frF,\sfP)$-martingale},
\end{align}
where $\frF = (\calF_t)_{t\ge 0}$ is the canonical filtration $\calF_t = \sigma(X_s:s\le t)$, and $L\colon B_b(E)\to B_b(E)$ is a bounded Markov generator such that $L^*\pi = 0$ for some stationary measure $\pi\in \calP(E)$.

We now write \eqref{eq:classical-martingale} in the form above. Set $\frX\coloneqq L^\infty(E,\pi)$ and $\frA\coloneqq L^\infty(\Omega,\sfP_\pi)$, which are both commutative von Neumann algebras and where
\[
	\sfP_\pi(A) = \int_E \sfP(x+A)\,\pi(\d x).
\]
Now, define the stochastic process
\[
	\frz_t\colon\frX\to (\frA,\mu),\qquad \frz_t(f) = f\circ X_t,
\]
with the state
\[
	\mu(G) \coloneqq \int_\Omega G(\omega)\,\sfP_\pi(\d\omega),\qquad G\in\frA,
\]
which makes $(\frA,\mu)$ a quantum probability space.
\[
	\sfE_t[\frz]
\]

and the $*$-homomorphism
\[
	\frz_t(f) - \frz_0(f) - \int_0^t \frz_{r-}(Lf)\dd r
\]
\[
	\upalpha_t(F) = F\circ \frs_t
\]
\[
	\upalpha_t\circ\frz_0(f) = \upalpha_t(f\circ X_0\otimes\mathbf{1}_\frY) = f\circ X_0\circ\frs_t(\omega)
\]
\[
	\upalpha_t \colon \calA {\times} \Omega \to \calA;\;\;\upalpha_t(M_f,\omega) = M_{f\circ X_t(\omega)}.
\]
The martingale identity \eqref{eq:classical-martingale} above can then be expressed as
\[
	\frm_t[M_f] \coloneqq \upalpha_t(M_f)- \upalpha_0(M_f) - \int_0^t \upalpha_{r-}(M_{Lf})\,\d r.
\]
Setting $\scrL M_f \coloneqq M_{Lf}$, the previous identity allows one to write \eqref{eq:classical-martingale} as
\[
	\frm_t[A] = \upalpha_t(A)- \upalpha_0(A) - \int_0^t \upalpha_{r-}(\scrL A)\,\d r,\qquad A\in \calA,
\]

thereby generalizing the martingale problem to a non-commutative setting.
\[
	\frm_t[A]-\frm_s[A] = \upalpha_t(A)- \upalpha_s(A) - \int_s^t \upalpha_{r-}(\scrL A)\,\d r
\]
 
\[
	\scrL_a = a^*[\bullet,a] + [a^*,\bullet]a
\]

\begin{example}[Diffusion]
Consider the formal example of the (possibly degenerate) diffusion process with the generator
\[
	Lf = \text{div}(A\nabla f) - A\nabla V\cdot \nabla f,\qquad f\in \calC_c^\infty(\bbR^d),
\]
i.e., it is the generator of the It\^o diffusion
\[
	\d X_t = -\sigma(X_t)\sigma^\top(X_t)\nabla V(X_t)\d t + \sigma(X_t)\circ\d B_t,
\]
with $A=\sigma\sigma^\top\in \bbR^{d\times d}$, $\sigma\in \bbR^{d\times m}$ 
\[
	a_i = \sum_{j=1}^d \sigma_{ji}\partial_j,\qquad a_i^* = -\sum_{j=1}^d\partial_j(\sigma_{ji}\,\bullet),\qquad i=1,\ldots,m.
\]
It is not difficult to see that $[a_i^*,M_f] = -[a_i,M_f]$. Moreover, for any $g\in C_c^\infty(\bbR^d)$,
\begin{align*}
	a_i^*[M_f,a_i]g &= \sum_{j=1}^d \partial_j(\sigma_{ji}g\sum_{k=1}^d\sigma_{ki}\partial_kf) \\
	&= \sum_{j=1}^d g\partial_j(\sigma_{ji}\sum_{k=1}^d\sigma_{ki}\partial_kf) + \sum_{j,k=1}^d \sigma_{ji}\sigma_{ki}\partial_j g \partial_kf \\
	&= \sum_{j=1}^d g\partial_j(\sigma_{ji}\sum_{k=1}^d\sigma_{ki}\partial_kf) - [M_f,a_i]a_ig.
\end{align*}
Consequently, we find that
\[
	\sum_{i=1}^m \scrL_{a_i}(M_f) = \sum_{i=1}^m\bigl(a_i^*[M_f,a_i] + [a_i^*,M_f]a_i\bigr) 
%	= \sum_{i=1}^m\sum_{j,k=1}^d \partial_j(\sigma_{ji}\sigma_{ki}\partial_kf) 
	= M_{\text{div}(A\nabla f)}
\]
Moreover, setting
\[
	H_i =  \sum_{j,k=1}^d \sigma_{ji}\sigma_{ki}\partial_j V \partial_k,\qquad i=1,\ldots,m,
\]
we have that
\[
	\sum_{i=1}^m [H_i,M_f] = M_{\sigma\sigma^\top\nabla V\cdot\nabla f}
\]
\[
	[a_i,M_V][a_i,\bullet]
\]
Therefore, we obtain
\[
	\scrL M_f = \sum_{i=1}^m \bigl([H_i,M_f] +\scrL_{a_i}M_f \bigr) = M_{Lf}\qquad \forall f\in \calC_c^\infty(\bbR^d).
\]

However, we know that the diffusion process constitutes a gradient flow with driving energy $\calF=\ent(\bullet|\pi)$, where $\pi=e^{-V}\leb$ is the invariant measure. Formally, one defines a state on $\calC_c^\infty(\bbR^d)$ by
\[
	\tau(M_f) \coloneqq \int_{\bbR^d} f\,\d \pi,\qquad f\in \calC_c^\infty(\bbR^d).
\]
\[
	\tau_\pi(M_fM_g) = \tau(M_fM_gM_\pi) = \tau(M_{fg\pi}) = \tau_\pi(M_gM_f)
\]
\[
	\Delta_\pi A = M_\pi A M_\pi^{-1} 
\]
\begin{align*}
	(\Delta_\pi V_j)(g) &= M_\pi V_j M_\pi^{-1}g = \pi \sum_{j=1}^d \sigma_{ji}\partial_j(\pi^{-1}g) - \sigma_{ji}\partial_jV (\pi^{-1}g) \\
	&= \sum_{j=1}^d \sigma_{ji}[g\partial_j V  + \partial_jg] - \sigma_{ji}\partial_jV g = 
\end{align*}
\[
	V_j = \sum_{j=1}^d \sigma_{ji}\partial_j - M_{\sigma_{ji}\partial_jV}
\]
\[
	\tau_\sigma(M_g\scrL M_f) = \hat\sigma(M_{gLf}) = \int_{\bbR^d} gLf\,\d \sigma = 
\]






\begin{align*}
	\scrL M_{e^{-V}}g &= \bigl(\text{div}(A\nabla e^{-V}) + A\nabla V\cdot \nabla e^{-V}\bigr)g \\
	&= \bigl(-\text{div}(A\nabla Ve^{-V}) - A\nabla V\cdot \nabla V e^{-V}\bigr)g
\end{align*}

\newpage



\[
	[a_i,M_f]g = \sum_{j=1}^d \sigma_{ji}\partial_j(fg) - \sum_{j=1}^d f\sigma_{ji}\partial_jg = \sum_{j=1}^d \sigma_{ji}g\partial_jf = M_{a_i(f)}g
\]
\[
	[a_i^*,M_f]g = -\sum_{j=1}^d\partial_j(\sigma_{ji}\, fg) + \sum_{j=1}^d f\partial_j(\sigma_{ji}\,g) = -\sum_{j=1}^d\sigma_{ji} \partial_jf g = -M_{a_i(f)}g = -[a_i^*,M_f]g
\]

Consequently, we find that
\[
	\scrL_{a_i}M_f = 
\]
\[
	[a_i,M_f]a_ig = \sum_{j,k=1}^d \sigma_{ji}\sigma_{ki}\partial_jf\partial_jg
\]

\[
	[a_i,M_V] = \sum_{j=1}^d \sigma_{ji}\partial_j V
\]
\[
	[a_i,[a_i,M_f]]=[a_i,M_{a_i(f)}] = M_{a_i(a_i(f))},
\]
where we used the fact that $[a_i,M_f] = M_{a_i(f)}$ for every $i=1,\ldots,d$.
\begin{align*}
	\sum_{i=1}^m a_i\circ a_i (f) &= \sum_{i=1}^m\sum_{j=1}^d \sigma_{ji}\partial_j\left(\sum_{k=1}^d \sigma_{ki}\partial_k f\right) = \sum_{i=1}^m\sum_{j,k=1}^d \sigma_{ji}\partial_j(\sigma_{ki}\partial_k f) \\
	&= \sum_{i=1}^m\sum_{j,k=1}^d \sigma_{ji}\partial_j(\sigma_{ki}\partial_k f)
\end{align*}
\begin{align*}
	\text{div}(\sigma\sigma^\top \nabla f) &= \sum_{j=1}^d\sum_{i=1}^m \sum_{k=1}^d \partial_j(\sigma_{ji}\sigma_{ki}\partial_k f) \\
	&= \sum_{j=1}^d\sum_{i=1}^m \sum_{k=1}^d \partial_j(\sigma_{ji}\sigma_{ki}\partial_k f) \\
\end{align*}



\end{example}

\begin{align*}
	\langle \psi|M_{Lf}|\psi\rangle &= \int (Lf)(x) |\psi(x)|^2\,\pi(\d x) \\
	&= -\int \sigma^2 \partial_xf \, \partial_x(\psi^2\pi)\,\d x - \int \sigma^2\partial_x f\, \partial_x V\,\psi^2\,\d\pi \\
	&= -2\int \sigma^2 \partial_x f \psi\partial_x\psi \,\d\pi \\
	&= -2 \int 
\end{align*}
\[
	a\psi = \sigma\partial_x \psi 
\]
\begin{align*}
	\langle \varphi|a\psi\rangle &= \int \varphi
	\, \sigma\, \partial_x\psi\,\d\pi  \\
	&= -\int \partial_x(\sigma\varphi \pi)\psi(x)\,\d x \\
	&= -\int \partial_x(\sigma \varphi)\psi\,\d \pi + \int \varphi \psi\, \sigma\partial_xV\,\d\pi \\
	&= \int \bigl[-\partial_x(\sigma\varphi)+\varphi \sigma\partial_x V\bigr]\psi\,\d\pi = \langle a^*\varphi|\psi\rangle
\end{align*}
\[
	(a^* M_fa\psi)(x) = -\partial_x(f\sigma^2\partial_x \psi) + f\sigma^2\partial_x \psi \partial_x V
\]
\[
	(a^* aM_f\psi)(x) = -\partial_x(\sigma^2\partial_x(f\psi)) + \sigma^2\partial_x(f\psi)\partial_x V
\]
\[
	(M_fa^* a\psi)(x) = -f\partial(\sigma^2\partial_x\psi) + \sigma^2f\partial_x\psi\partial_x V
\]
\begin{align*}
	\int \psi(x)(a^* M_fa\psi)(x)\,\d\pi &= \int f|\sigma\partial_x\psi|^2\,\d\pi 
\end{align*}

%\begin{align*}
%	\langle \psi|\scrL_a M_f|\psi\rangle 
%	&= \int f(x)|(a\psi)(x)|^2\,\pi(\d x) - \langle a^* a\psi|M_f\psi\rangle  
%\end{align*}
%
%
%\begin{align*}
%	(M_{Lf}\psi)(x) &= \int Lf\psi =  \\
%	&= \iint f(y) \kappa(x,\d y)\psi(x)\pi(\d x) - \iint f(x) \kappa(x,\d y)\psi(x)\pi(\d x) \\
%	&= \upalphac{1}{2} \iint f(y) \psi(x)\kappa(x,\d y)\pi(\d x) + \upalphac{1}{2} \iint f(y) \psi(x)\kappa(y,\d x)\pi(\d y) \\
%	&- \upalphac{1}{2}\iint f(x) \psi(x)\kappa(x,\d y)\pi(\d x) - \upalphac{1}{2}\iint f(x) \psi(x)\kappa(y,\d x)\pi(\d y)
%\end{align*}
%\[
%	\kappa(x,\d y)\psi(x)
%\]
%\begin{align*}
%	\int ([a,M_f]\psi)(x)\pi(\d x) &= \int \bigl(aM_f\psi - M_fa\psi\bigr)\,\pi(\d x) \\
%	&= \kappa(x,\calX)\psi(x)
%\end{align*}
%\[
%	(aM_f)g(x) = \int [f(y)g(y)- f(x)g(x)]\kappa (x,\d y)
%\]
%\begin{align*}
%	[a,M_f]g(x) &= \int [f(y)g(y)- f(x)g(x)]\kappa (x,\d y) - f(x)\int [g(y)- g(x)]\kappa (x,\d y) \\
%	&= \int [f(y)-f(x)]g(y)\kappa(x,\d y)
%\end{align*}
%
%\begin{example} 
%
%$L^2_\bbR(\calX,\pi)$, $\vartheta(\d x\d y)=\kappa(x,\d y)\pi(\d x)$ is symmetric.
%\[
%	af(x) = \int [f(y)- f(x)]\kappa (x,\d y)
%\]
%\end{example}
%\begin{align*}
%	a^*[a,M_f]g(x) &= - \int \left[\int [f(z)-f(y)]g(z)\kappa(y,\d z)-\int [f(z)-f(x)]g(z)\kappa(x,\d z)\right] \kappa(x,\d y)\\
%	&= -\int \left[\int f(z)g(z)\kappa(y,\d z) - \int f(z)g(z)\kappa(x,\d z)\right] \kappa(x,\d y) \\
%	& + \int f(y) \int g(z)\kappa(y,\d z) - f()
%\end{align*}
%
%\[
%	a^* = -a
%\]
%
%
%\begin{align*}
%	\iint f(x,y)(a\psi)(x,y)\pi(\d x)\pi(\d y) &= \iint f(x,y)\psi(x)\sqrt{\kappa(x,y)}\pi(\d x)\pi(\d y) \\
%	&= \int \psi(x)\left(\int f(x,y)\sqrt{\kappa(x,y)}\pi(\d y)\right)\pi(\d x) = \langle \psi|a^* f\rangle 
%\end{align*}
%
%\begin{align*}
%	\langle \psi|\scrL_a M_f|\psi\rangle &= \upalphac{1}{2}\langle \psi| a^*[M_f,a]\psi\rangle + \upalphac{1}{2}\langle\psi|[a^*,M_f]a\psi\rangle \\
%	&= \upalphac{1}{2}\langle a\psi|M_f|a\psi\rangle - \upalphac{1}{2}\langle \psi|a^* aM_f\psi\rangle + \upalphac{1}{2}\langle a\psi| M_f|a\psi\rangle - \upalphac{1}{2}\langle M_f\psi|a^* a\psi\rangle \\
%	&= \int f(x)|(a\psi)(x)|^2\,\pi(\d x) - \langle a^* a\psi|M_f\psi\rangle  
%\end{align*}
%\[
%	\int (a\psi)(x) a(f\psi)(x)\,\pi(\d x)
%\]
%\[
%	(a^* (M_f\otimes 1) a\psi)(x) = \int f(x)\sqrt{\kappa(x,y)}\psi(x)\sqrt{\kappa(x,y)}\pi(\d y)
%\]
%\[
%	(a^* a\psi)(x) = \int \psi(x)\kappa(x,y)\pi(\d y)
%\]
%
%
%

\subsection{Stochastic processes on matrix algebras}

In this section, we consider $\frX=\textsf{Mat}(\bbC,n)$. Let $a\in \calO(\frX)$ be an observable on $\frX$ and set
\[
	\scrL x = a^*[x,a] + [a^*,x]a,\qquad x\in \frX.
\]
We then consider the solution of the
\[
	\d \scrU_t = \bigl(-a^* a\,\d t + i\sqrt{2}a\,\d B_t\bigr)\scrU_t,
\]
which, due to $a=a^*$, can be explicitly expressed as
\[
	\scrU_t(x,e(f)) \coloneqq \exp(i\sqrt{2} a B_t(\omega))x,\qquad t\in\bbT,
\]
which is a $*$-automorphism on $\frX$, which extends to an isometry 
\[
	\scrU_t\colon \frX\otimes \calA
\]
\[
	B\in \calA
\]

\[
	\frz_t(x) - \frz_0(x) - \int_0^t \frz_r(\scrL x)\dd r
\]


\subsection{Stochastic processes on the unitary group}

\subsubsection{The unitary group and its Lie structure}

The unitary group is defined as
\[
U(n) = \bigl\{ U \in \mathbb C^{n\times n} : U^* U = I \bigr\}.
\]
It is a compact Lie group with Lie algebra (tangent space at identity)
\[
	\calU(n) = \sfT_IU(n) = \bigl\{ \fra \in \mathbb C^{n\times n} : \fra^* = -\fra \bigr\},
\]
i.e., the space of skew-Hermitian matrices. A vector field $\fra\colon U(n)\to \calU(n)$ is \emph{left--invariant} if $\fra(U) = U \fra(I)\in\calU(n)$ for all $U\in U(n)$.

The Lie algebra $\calU(n)$ can be equipped with a real inner product given by the Hilbert-Schmidt inner product
\[
	\langle \fra, \frb \rangle_{\calU(n)} = -\tr(\fra^* \frb) = \Re\, \tr(\fra \frb^*),
\]
which is positive-definite on $\calU(n)$. The associated norm is then
\[
	|\fra|_{\calU(n)}^2 = \langle \fra, \fra \rangle_{\calU(n)} = \mathrm{Tr}(\fra \fra^*).
\]

\subsubsection{Unitary-valued Brownian motion (rank-$r$ noise)}

Let $\fra_1,\dots,\fra_r \in \calU(n)$ be orthonormal under the inner product $\langle\cdot,\cdot\rangle_{\calU(n)}$, where $1\le r\le n^2$.   Further, let $\beta_t^1,\dots,\beta_t^r$ be independent standard real Brownian motions.

Define the $\calU(n)$--valued (possibly degenerate) Brownian driver
\[
	\frw_t = \sum\nolimits_j \fra_j\,\beta_t^j .
\]
If $r=n^2$, the driver is elliptic (non-degenerate).  
If $r< n^2$, the covariance has rank $r$ and the process is \emph{degenerate} (hypoelliptic), exploring only the connected subgroup
\[
	H = \exp(\calH)\subset U(n),\quad \calH = \mathrm{Lie}\{\fra_1,\dots,\fra_r\}\subset\calU(n).
\]

The intrinsic $U(n)$-valued diffusion process solves the Stratonovich SDE
\[
	dU_t = U_t \circ d\frw_t, \qquad U_0 = I,
\]
or in components
\[
	dU_t = \sum\nolimits_j U_t \fra_j \circ d\beta_t^j.
\]
The matrix-valued quadratic variation of $\frw$ is given by
\[
	d[\frw] = \sum\nolimits_j \fra_j\otimes \fra_j\,dt.
\]

In It\^o form, the equivalent SDE reads
\[
	dU_t = U_t\bigl( d\frw_t + \frl\,dt\bigr),\qquad \frl := \frac{1}{2}\sum\nolimits_j \fra_j^2,
\]
where $\frl$ is the Laplace-Beltrami generator associated with the left-invariant connection.

In the following, we denote its law on path space $C([0,T];U(n))$ by $\sfP$.

\subsubsection{Girsanov transform on $U(n)$ (Karandikar type)}

Let $\fru\in L^2((0,T);\calU(n))$ be adapted and square-integrable.

Define the drifted group motion by
\[
	dU_t^\fru = U_t^\fru \circ (d\frw_t + \fru_t\,dt),\qquad U_0^\fru = I.
\]
Assuming a Novikov condition in the active noise directions,
\[
\mathbb E\left[ \exp\left( \frac12\int_0^T |\fru_s|^2 \,ds \right) \right] < +\infty,
\]
the process defined below is a true martingale.

Decompose the drift against the active algebra frame:
\[
	\fru_t = \sum\nolimits_j \fra_j\,v_t^j,\qquad v=(v^1,\dots,v^r)\in L^2((0,T);\bbR^r).
\]
Define the exponential Cameron-Martin density process
\begin{align*}
	M_T &= \exp\left( \int_0^T \langle \fru_t, d\frw_t\rangle - \frac12\sum\nolimits_j\int_0^T \langle \fru_t,\fra_j \otimes \fra_j \,\fru_t\rangle\,  dt \right) \\
	&= \exp\left( \sum\nolimits_j \int_0^T v_t^j\, d\beta_t^j - \frac12\sum\nolimits_j \int_0^T |v_t^j|^2 dt \right).
\end{align*}
This expression depends only on the directions with nonzero quadratic variation. It is the standard Cameron-Martin shift density on $\mathbb{R}^r$ lifted to the group via left multiplication.

We then define a new path measure on the same probability space by
\[
	\sfQ = M_T\,\sfP.
\]
Under $\sfQ$:
\begin{enumerate}[label=(\roman*)]
	\item The shifted drivers
\[
\beta_t^{j,\sfQ} := \beta_t^j - \int_0^t v_s^j\, ds
\]
are Brownian motions again on $\bbR$.

\item The algebra noise
\[
	d\frw_t^{\sfQ} := d\frw_t - \fru_s\, dt = \sum\nolimits_j \fra_j\,d\beta_t^{j,\sfQ}
\]
is a $\calU(n)$-valued Brownian motion again with same covariance 
\[
	d[\frw^\sfQ] = d[\frw] = \sum\nolimits_j \fra_j\otimes\fra_j\,dt.
\]

\item The group paths satisfy the same \emph{driftless} intrinsic SDE
\[
	dU_t^\fru = U_t^\fru \circ dW_t^\sfQ = \sum\nolimits_j U_t^\fru \fra_j \circ d\beta_t^{j,\sfQ},
\]
so $U_t^\fru$ is an intrinsic Brownian motion on the same symmetry group $H$.
\end{enumerate}

Hence, $M_T$ is the Radon-Nikodym derivative between the laws of the drifted and driftless group diffusions on path space.

\subsubsection{Induced Girsanov on pure states}

Rank-1 projections evolve by the conjugation map
\[
P_t=U_tP_0U_t^*,\qquad P_0=\psi_0\psi_0^*.
\]
This gives a valid reference probability on paths of $C([0,T];\mathcal M)$, where
\[
\mathcal M = \{ V P_0 V^* : V\in H \}\subset \mathcal P_1(n)\simeq\mathbb{CP}^{n-1}
\]
is the homogeneous orbit that the degenerate noise can reach.

Linearising conjugation and applying the Stratonovich chain rule yields the SDE
\[
	dP_t^\fru = [\circ dW_t^\sfQ,\,P_t^\fru] = \sum\nolimits_j [\fra_j,P_t^\fru] \circ d\beta_t^{j,\sfQ},
\]
so the pure-state path law remains Brownian on the same orbit manifold $\mathcal M$.

If a drift $\fru$ has components outside $\mathrm{span}\{\fra_j\}$, only the projection of the drift onto that span can be removed. Directions with zero quadratic variation cannot absorb drifts.

\[
	\int F(U_tP_0U_t^*)\,\sfP(dU) =  
\]


\newpage


\subsection{Dilation of semigroups (Sz.Nagy)}

\subsection{Stochastic dilations: How to make a heat bath}

\subsection{Interpretation of LDPs on unitary evolutions}

The task of this subsection is to reinterpret the large deviations of empirical measures in a functional analytic framework.

Consider a family of iid $E$-valued random variables $(X_i)_{i\in\bbN}$ with $\sigma = \text{Law}X_1$. Setting $\Omega=E^\bbN$ and $\sfR=\sigma^{\otimes\bbN}$, we see that $\text{Law}\, (X_i)_{i\in\bbN} =\sfR$. Moreover, we can disregard the initial probability space for $X_i$ and consider instead the canonical random variable on $\Omega$.

For every $f\in \frX\coloneqq\calC_b(E)$ and $n\in\bbN$, we define the noncommutative random variable
\[
	\frX\ni f\mapsto \frz^n(f) = \frac{1}{n}\sum_{i=1}^n f(X_i) \in \frA\coloneqq L^\infty(\Omega,\sfR),
\]
where $\frA$ is equipped with the state $\mu(G)=\sfE_\sfR[G]$, $G\in \frA$.
\[
	\mu(\frz^n(f)) = \frac{1}{n}\sum_{i=1}^n \sfE_\sfR[f(X_i)] = \frac{1}{n}\sum_{i=1}^n \int_E f\dd\sigma = \int_E f\dd\sigma = \mu(\frz_1(f)).
\]
\begin{align*}
	\sfR(\sup_{f}\{\mu^n(f)-\nu(f)\}>\varepsilon) &= \sfR(n\mu^n(f)>n(\nu(f)+\varepsilon)) \\
	&= \sfR(e^{n\mu^n(f)}>e^{n(\nu(f)+\varepsilon)}) \le e^{-n(\nu(f)+\varepsilon)}\int_\Omega e^{n\mu^n(f)}\dd\sfR
\end{align*}
\[
	\frz \colon \frX\to (\frX\otimes\frC,\tr\otimes\mu)
\]
\[
	\frA = \frX\otimes\frC^{\otimes\bbZ},\quad \upvarphi = \tr\otimes \mu^{\otimes\bbZ}
\]
Let $X_i$
\[
	\frz_i(x) = x\otimes c_i,\qquad c_i = \cdots 1\otimes c
\]
 
\[
	\frz^n = \frac{1}{n}\sum_{i=1}^n \frz_i,\qquad \frz_i = \cdots\otimes\mathbbm{1}\otimes\frz\otimes\mathbbm{1}\otimes\cdots
\]

\[
	\upvarphi(\frz^n(x)) = \frac{1}{n}\sum_{i=1}^n \upvarphi(\frz_i(x)) = \upvarphi(\frz_1(x)) = \mu(\frz(x))
\]
\[
	\frz_i(x) = \mathbbm{1}\otimes \cdots \otimes\mathbbm{1}\otimes\frz(x)\otimes \mathbbm{1}\otimes\cdots\otimes\mathbbm{1}
\]
\[
	\frz^n(x)\in \frA\cong \frX\otimes\frY
\]
\[
	\mu = \tr\nolimits \otimes \mu_\frC^{\otimes\bbN}
\]
\[
	\mu(\mathbf{1}_A(\frz^n(x)-\fry_1(x))) = \mu_\frC^{\otimes\bbN}\bigl(\tr[\mathbf{1}_A(\frz^n(x)-\fry(x)\otimes\mathbbm{1}_{\frC^{\otimes\bbN}})]\bigr) 
\]


\[
	\sup_{f}\{\mu^n(f)-\nu(f)\}>\varepsilon\;\;\Rightarrow\;\; \mu^n(f^\delta) - \nu(f^\delta)>\varepsilon - \delta
\]
\begin{align*}
	\sfR(\mu^n(f)-\nu(f) > \varepsilon)
	&= \sfR(e^{n\mu^n(f^\delta)} > e^{n(\nu(f^\delta) +\varepsilon - \delta)}) \\
	&\le e^{-n(\nu(f^\delta) +\varepsilon - \delta)}\sfE_\sfR[e^{n\mu^n(f^\delta)}]
\end{align*}

\begin{align*}
	-\frac{1}{n}\log \sfR(\mu^n(f)-\mu(f)>\varepsilon) &\ge \mu(f)+\varepsilon -\frac{1}{n}\log\int_\Omega e^{n\mu^n(f)}\dd\sfR \\
	&= \mu(f) + \varepsilon - \log \int_E e^{f}\d\mu
\end{align*}
\[
	-\frac{1}{n}\log \sfR(\mu^n(f)-\mu(f)>\varepsilon)\ge 
\]
Taking the sup
\begin{align*}
	\int_\Omega e^{n\mu^n(f)}\dd\sfR &= \int_\Omega \prod_{i=1}^n e^{f(X_i)}\dd\sfR \\
	&= \int_{E^N} \prod_{i=1}^n e^{f(x_i)} \mu(\d x_1)\cdots\mu(\d x_n) = \left(\int_E e^{f(x)}\mu(\d x)\right)^n
\end{align*}


\subsection{LDP via G\"artner-Ellis revisited}
Let $\Omega=\calC(I;\frh)$ be the space of $\frh$-valued continuous path and consider the canonical process $(X_t)_{t\in I}$ defined by $X_t(\omega):=\omega(t)$ for all $t\in I$. Further, let $\calF = \cup_{t\ge 0}\calF_t$, where the filtration $\frF:=\{\calF_t:=\sigma(X_s:0\le s\le t)\}_{t\in I}$ is generated by $(X_t)_{t\in I}$. Throughout, we consider a reference path measure $\sfR\in\calP(\Omega)$ such that $X$ admits the $\sfR$-semimartingale decomposition
\[
	X = X_0 + B^\sfR + M^\sfR\qquad\text{$\sfR$-almost surely},
\]
where $B^\sfR$ is an adapted process with absolutely continuous  sample paths $\sfR$-almost surely, and $M^\sfR$ is an $\sfR$-martingale. For simplicity of presentation, we assume that the quadratic variation of $M^\sfR$ is an $\calS_+(\frX)$-valued random measure on $I$ with Lebesgue dentisy, i.e.,
\[
	\frac{\dd [M^\sfR,M^\sfR]}{\dd\lambda}(t) = \sfa_t(X_t)\in \calS_+(\frh)\qquad \text{$\sfR$-almost surely},
\]
and also that $t\mapsto \sfa_t(X_t)$ is an $\frF$-adapted process.





and has the quadratic variation $\langle M^\sfR[\varphi]\rangle$ for every smooth function $\varphi$.

Let $X^1,X^2,\ldots$ be independent copies of 

$(X^1,\ldots,X^n)$

\[
	\frac{\dd \sfQ^{n}}{\dd\sfP^{\otimes n}} = \exp\left( \sum_{i=1}^n\int_I \langle\beta_t^i(X_t),\dd  M_t^\sfR\rangle - \frac{1}{2}\sum_{i=1}^n\int_I \langle \beta_t^i(X_t),\sfa_t(X_t)\beta_t^i(X_t)\rangle\dd t \right) =: Z^{(n)}
\]
\[
	Y_t^{(n)} := \sum_{i=1}^n\int_0^t \langle\beta_s^i(X_s),\dd  M_s^\sfR\rangle - \frac{1}{2}\sum_{i=1}^n\int_0^t \langle \beta_s^i(X_s),\sfa_s(X_s)\beta_s^i(X_s)\rangle\dd s
\]
\begin{align*}
	\dd Z_t^{(n)} = \dd e^{Y_t^{(n)}} = Z_t^{(n)}\left(\dd Y_t^{(n)} + \frac{1}{2} \dd [Y^{(n)},Y^{(n)}]_t\right) = \sum_{i=1}^n  Z_t^{(n)} \langle\beta_t^i(X_t),\dd  M_t^\sfR\rangle
\end{align*}


Define $\rho^n:=\frac{1}{n}\sum_{i=1}^n X^i$. Then,
\[
	\dd \rho^n = \frac{1}{n}\sum_{i=1}^n \dd X^i_t = \frac{1}{n}\sum_{i=1}^n b_t^\sfR(X^i_t)\dd t + \frac{1}{n}\sum_{i=1}^n\dd M^{\sfR,i}_t
\]


\begin{align*}
	\sfP^{\otimes n}\left(\frac{1}{n}\sum_{i=1}^n X^i \in A\right) &= \int_A\,e^{-\int_I \langle\beta_t^i(X_t),\dd  M_t^\sfR\rangle + \frac{1}{2}\int_I \langle \beta_t^i(X_t),\sfa_t(X_t)\beta_t^i(X_t)\rangle\dd t}\dd\sfQ^{(n)} \\
	&= \int_A\,e^{-\int_I \langle\beta_t^i(X_t),\dd  M_t^\sfR\rangle + \frac{1}{2}\int_I \langle \beta_t^i(X_t),\sfa_t(X_t)\beta_t^i(X_t)\rangle\dd t}\dd\sfQ^{(n)}
\end{align*}

\[
	A:= \{ \rho : \partial_t\rho + \ddiv j =  L\rho\}
\]
\begin{align*}
	D_{\beta^i} \exp\left( \int_I \langle\beta_t^i(X_t),\dd  M_t^\sfR\rangle + \frac{1}{2}\int_I \langle \beta_t^i(X_t),\sfa_t(X_t)\beta_t^i(X_t)\rangle\dd t \right)[h^i]
\end{align*}

\begin{align*}
	\int_I \langle\beta_t, j_t\rangle\dd t -  \log \int_A Z^{(n)}\dd\sfP^{\otimes n}
\end{align*}

\subsubsection{Example}

We consider the Hilbert space $\frh = \bbM$ equipped with the scalar product $\langle A,B\rangle = \tr[A^* B]$. Consider the stochastic process
\[
	P_t = P_0 + \int_0^t L P_s\dd s + \int_0^t \dnabla P_s\dd W_s ,
\]
where $\dnabla P = [i\sigma_X,P]$ and $LP = (1/2)[i\sigma_X,[i\sigma_X,P]]$. In other words,
\[
	B_t^\sfR = \int_0^t L P_s\dd s,\qquad M_t^\sfR = \int_0^t \dnabla P_s\dd W_s,
\]
where $M^\sfR$ has the quadratic variation
\[
	\sfa(P) = \langle(\dnabla P)^*,\bullet\,\rangle\,\dnabla P\qquad\text{for every $P\in \frh$}.
\]
Let $P^1,P^2,\ldots$ be independent copies and define $\rho^n:=\frac{1}{n}\sum_{i=1}^n P^i$. Then,
\[
	\dd \rho_t^n =  L\rho_t^n\dd t + \frac{1}{n}\sum_{i=1}^n\dnabla P^i_t \dd W^i_t = L\rho_t^n\dd t + \frac{1}{n}\dd M_t^{\sfR,n}.
\]
\[
	\frac{1}{n}\sum_{i=1}^n\dnabla P^i_t \dd W^i_t = \frac{1}{n}\sum_{i=1}^n \Bigl(\dnabla P^i_t \dd W^i_t - \sfa(P_t^i)H_t\dd t\Bigr) + \frac{1}{n}\sum_{i=1}^n \sfa(P_t^i)H_t\dd t
\]
\[
	\dd \rho^n = L\rho^n\dd t + \frac{1}{n}\sum_{i=1}^n\sfa(P_t^i)H_t\dd t + \frac{1}{n}\dd M_t^{\sfQ,n}
\]
For each $n\in\bbN$, we set
\[
	\frac{\dd \sfQ^{n}}{\dd\sfR^{\otimes n}} = \exp\left( \sum_{i=1}^n\int_I \langle \dnabla P_t^i,H_t\rangle \dd W_t^i - \frac{1}{2}\sum_{i=1}^n\int_I \langle H_t,\sfa(P_t^i) H_t\rangle\dd t \right) =: Z^{(n)}.
\]
Then, for curves $\eta$ satisfying $\partial_t\eta + \ddiv J^\eta = L\eta$ for some $J^\eta$, we find that
\[
	J^\eta = \sum \langle \sigma_j,J^\eta\rangle\sigma_j
\]
Suppose
\[
	-\ddiv = \sum_{j\in J} \alpha_j[i\sigma_j,\bullet],
\]
then
\[
	-\ddiv J^\eta = \sum_{j,k} \alpha_j \langle \sigma_k,J^\eta\rangle[i\sigma_j,\sigma_k] = \sum_{j,k\ne j} \alpha_j \langle \sigma_k,J^\eta\rangle[i\sigma_j,\sigma_k] \in \text{span}\bigl\{ [i\sigma_j,\sigma_k]: j\in J,k\ne j\bigr\}
\]



\begin{align*}
	\sfR^{\otimes n}\bigl(\rho^n \in A\bigr) &= \int_A\exp\biggl( -\sum_{i=1}^n\int_I \langle \dnabla P_t^i,H_t\rangle \dd W_t^i + \frac{1}{2}\sum_{i=1}^n\int_I \langle H_t,\sfa(P_t^i) H_t\rangle\dd t \biggr)\dd\sfQ^{(n)} \\
	&= \int_A\exp\biggl( -\int_I \langle \dd M_t^{\sfR,n},H_t\rangle + \frac{1}{2}\sum_{i=1}^n\int_I \langle H_t,\sfa(P_t^i) H_t\rangle\dd t \biggr)\dd\sfQ^{(n)} \\
	&= \int_A\exp\biggl( -\int_I \langle \dd M_t^{\sfQ,n},H_t\rangle - \frac{1}{2}\sum_{i=1}^n\int_I \langle H_t,\sfa(P_t^i) H_t\rangle\dd t \biggr)\dd\sfQ^{(n)}
\end{align*}

\begin{align*}
	H^\eta\in \argmin_H \left\{\sfE_\sfQ\biggl[ \frac{1}{2}\int_I \langle H_t,\sfa(P_t) H_t\rangle\dd t\biggr] - \int_I \langle H_t, \ddiv J_t^\eta\rangle\dd t \right\}
\end{align*}
\begin{align*}
	\sfE_\sfQ\biggl[ \int_I \bigl(\tr[P_t \dnabla H_t]\bigr)^2\dd t\biggr] = \int_I \sfE_\sfQ\Bigl[\bigl(\tr[P_t \dnabla H_t]\bigr)^2\Bigr]\dd t 
\end{align*}
\[
	\tr[\rho_t^\sfQ \dnabla H_t] = \sum \lambda_i(t) \langle v_i(t), \dnabla H_t v_i(t)\rangle \ge \gamma\sum \lambda_i(t)
\]
\[
	\sfE_\sfR\biggl[ \int_I \langle G_t,\sfa(P_t) H_t\rangle\dd t\biggr] = \int_I \langle G_t,\ddiv J_t\rangle\dd t 
\]

\begin{align*}
	\int_I \langle H_t,\ddiv J_t\rangle\dd t - \log \sfE_\sfR\biggl[\exp\biggl( \int_I \langle H_t, \dd M_t^{\sfR,n}\rangle - \frac{1}{2}\sum_{i=1}^n\int_I \langle H_t,\sfa(P_t^i) H_t\rangle\dd t\biggr)\biggr]
\end{align*}

\begin{align*}
	\int_I \langle G_t,\ddiv J_t\rangle\dd t &= -\sfE_{\sfR^{\otimes n}}\biggl[ \biggl( \int_I \langle G_t, \dd M_t^{\sfR,n}\rangle - \sum_{i=1}^n\int_I \langle G_t,\sfa(P_t^i) H_t\rangle\dd t\biggr)\exp\biggl( \int_I \langle H_t, \dd M_t^{\sfR,n}\rangle - \frac{1}{2}\sum_{i=1}^n\int_I \langle H_t,\sfa(P_t^i) H_t\rangle\dd t\biggr)\biggr]
\end{align*}

Let $Y_t^{(n)} = \sum_{i=1}^n\int_0^t \langle \dnabla H_s, \dnabla P_s^i\rangle \dd W_s^i$ and $Z_t^{(n)}:=\exp(Y_t^{(n)})$. Then,
\begin{align*}
	\dd Z_t^{(n)} = Z_t^{(n)} \biggl(\sum_{i=1}^n \langle \dnabla H_t, \dnabla P_t^i\rangle \dd W_t^i + \frac{1}{2} \sum_{i=1}^n \langle \dnabla H_t,\sfa(P_t) \dnabla H_t\rangle\dd t  \biggr)
\end{align*}
\[
	Z_t^{(n)} = 1 + \sum_{i=1}^n \int_0^t  Z_s^{(n)}\langle \dnabla H_s, \dnabla P_s^i\rangle \dd W_s^i + \frac{1}{2} \sum_{i=1}^n \int_0^t Z_s^{(n)}\langle \dnabla H_s,\sfa(P_s) \dnabla H_s\rangle\dd s 
\]

\newpage
Let $\varrho\in\calS(\calH)$ be state and $(\varrho^n)_{n\in\bbN}\subset\calS(\calH)$ be a sequence of states. We say that $\varrho^n\to \varrho$ setwise if
\[
	\varrho^n(E)\to \varrho(E)\quad\text{for all projections $E\in\calP(\calH)$.}
\]
Further, let $(\frz_t)_{t\in\bbT}$ be a stochastic process in the sense of Definition~\ref{def:random-variable} with
\[
	\frz_t\colon \frX\to (\frX\otimes\frC,\varphi),\quad \varphi = \varrho\otimes\mu.
\]
\[
	\frz_t(x)=\fr\alpha_t^* x \fr\alpha_t
\]
and $\frz_0(x) = x\otimes 1$, $x\in\frX$. By construction, we have that
\[
	\varphi(\frz_0(x)) = \varphi(x\otimes 1) = \varrho(x)\mu(1)=\varrho(x)\quad\forall\,x\in\frX.
\]

We suppose that 



\[
	\calS(\frX\otimes \frC^{\otimes n})\to \calS(\frX\otimes\frC)
\]
\[
	\mu_\varphi = \frac{1}{n}\sum_{i=1}^n\varphi_i = \frac{1}{n}\sum_{i=1}^n \rho\otimes \varphi
\]
\[
	\x_i:\frC^{\otimes n}\to \frC;\quad \x_i(\omega_1,\ldots,\omega_n)=\omega_i
\]
\[
	\delta_{\x_i}(E)
\]

\subsection{Groups to algebras}

Let $\bbT$ be the 1-dimensional torus. Further, let $\calH$ be a Hilbert space and $\sfh\in \calO(\calH)$ be an observable on $\calH$. Consider the unitary $\fru\colon \bbT\to \calU(\calH)$ such that $\fru(\exp(tX))=e^{it \sfh}$.
\[
	\frac{\d}{\d t}f(\exp(tX))\big|_{t=0} = X(f)
\]
\[
	\d X_t = 
\]

\[
	\calB(\calH) \to L^\infty(\bbT;\calB(\calH));\qquad a\mapsto \fru^*(z)a\fru(z)
\]



Let $\rho_0\in \calS(\calH)$ be a given fixed state
\begin{align*}
	\varrho_\mu(x) &= \int_\bbT \tr[\fru^*(z) x \fru(z)\varrho_0]\,\mu(\d z) = \int_\bbT f_{\rho_0,x}(z)\,\mu(\d z)\\
	&= \tr\left[x\int_\bbT \fru(z)\varrho_0\fru^*(z)\,\mu(\d z) \right] = \tr[x\varrho_\mu]
\end{align*}

Consider iid random variables $(Z^i)_{i\in\bbN}$, $Z^i\sim\mu\in\calP(\bbT)$ and the corresponding sequence of empirical measures
\[
	\mu^n = \frac{1}{n}\sum_{i=1}^n \delta_{Z^i},\qquad \varrho_{\mu^n}(x)=\frac{1}{n}\sum_{i=1}^n \fru(Z^i)\varrho_0\fru^*(Z^i)
\]


\begin{align*}
	\frac{1}{n}\log\mathbb{E}[e^{n\langle f,\mu^n\rangle}] &= \frac{1}{n}\log \mathbb{E} \left[ \exp\left(\sum_{i=1}^n f(Z^i)\right)\right] \\
	&= \log \left(\int_\bbT e^{f(z)}\mu(\d z)\right)
\end{align*}
\begin{align*}
	\sup_{f=f_{\varrho_0,x}} \left\{\langle f,\nu\rangle-\log \int_\bbT e^{f(z)}\mu(\d z)\right\} = \sup_{x\in\calO(\calH)} \left\{\tr[x\varrho_\nu]-\log \int_\bbT e^{\tr[\fru^*(z)x \fru(z)\varrho_0]}\mu(\d z)\right\}
\end{align*}
\begin{align*}
	&\sum_{j,k} x_{jk}\tr[E_{jk}\varrho_\nu] - \log\int_\bbT e^{\sum_{j,k}x_{jk}\tr[\fru^*(z)E_{jk}\fru(z)\varrho_0]}\mu(\d z) \\
	&=\sum_{j,k} x_{jk}\tr[E_{jk}\varrho_\nu] - \log\int_\bbT \prod_{j,k}e^{x_{jk}\tr[\fru^*(z)E_{jk}\fru(z)\varrho_0]}\mu(\d z)
\end{align*}
\[
	\tr[E_{jk}\varrho_\nu] = \frac{1}{c} \int_\bbT \tr[\fru^*(z)E_{jk}\fru(z)\varrho_0] e^{\sum_{j,k}x_{jk}\tr[\fru^*(z)E_{jk}\fru(z)\varrho_0]}\mu(\d z)
\]
\[
	b = \frac{1}{c}\int_\bbT v(z)e^{\langle\x,v(z)\rangle}\mu(\d z)
\]
Consider the map
\[
	\x\mapsto \Lambda(\x)=\int_\bbT v(z)e^{\langle\x,v(z)\rangle}\mu(\d z)
\]
\[
	\langle\Lambda(\x)-\Lambda(\y),\x-\y\rangle = \int_0^1 \langle D\Lambda((1-\tau)\x+\tau\y)[\x-\y],\x-\y\rangle\dd\tau \ge 0
\]
\[
	\frac{\d}{\d\lambda}\int_\bbT v(z)e^{\langle\lambda,v(z)\rangle}\mu(\d z) = \int_\bbT v(z)\otimes v(z)e^{\langle\lambda,v(z)\rangle}\mu(\d z)
\]

\[
	\tr[E_{jk}\fru(z)\varrho_0\fru^*(z)] = \langle \psi_k| \fru(z)\varrho_0\fru^*(z)|\psi_j\rangle = \sum_i \lambda_i \langle \psi_k| \fru(z)|\psi_i\rangle\langle\psi_i|\fru^*(z)|\psi_j\rangle
\]
If $\varrho_0=\mathbbm{1}$, then
\[
	\tr[E_{jk}\fru(z)\varrho_0\fru^*(z)] =\delta_{kj}.
\]
Hence,
\[
	\tr[E_{jk}\varrho_\nu] = \frac{1}{c}\delta_{kj} e^{\sum_j x_{j}}
\]







