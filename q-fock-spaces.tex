\section{Quantum Noise and Fock Space}
\label{sec:quantum-noise}

As shown in the previous lecture, stochastic Schr\"odinger equations driven by classical noise provide a useful class of stochastic unravellings of Lindblad dynamics. However, this approach is fundamentally limited: it produces only Lindblad generators with \emph{self-adjoint} jump operators. To describe general irreversible quantum dynamics---including spontaneous emission, particle loss, and counting processes---we must move beyond classical noise and introduce \emph{quantum noise}.

Recall that in Section~\ref{sec:open-quantum}, the stochastic dilation $\oba_t = U_t\rho_0 U_t^\dagger$, led, after taking expectations, to a Lindblad equation with Hermitian noise operators $L_j^\dagger = L_j$. While such Lindblad operators model dephasing and diffusion-type noise, they cannot describe dissipative processes such as amplitude damping, where the jump operator is non-Hermitian. An important example is the jump operator $L = \sigma_- = \ket{0}\bra{1}$ on a single qubit describing \emph{spontaneous emission} due to black body radiation.

From a physical perspective, this reflects the fact that classical noise models only randomize \emph{phases} or \emph{energies}. Truly quantum processes involve the exchange of quanta with an environment, and therefore require a non-commutative noise model.

\subsection{Unitary dilations and infinite environments}
A guiding principle in the theory of open quantum systems is that irreversibility arises from neglecting environmental degrees of freedom. Concretely, one considers a unitary evolution
\[
	U_t : \calH_{\text{sys}} {\otimes} \calH_{\text{env}} \to \calH_{\text{sys}}{\otimes} \calH_{\text{env}}
\]
on a composite Hilbert space consisting of the system $\calH_{\text{sys}}$ and an environment $\calH_{\text{env}}$. The reduced evolution of the system state is then obtained by tracing out the environment,
\[
	\rho_t = \tr\nolimits_\calK \bigl[U_t (\rho_0 {\otimes} \rho_{\text{env}}) U_t^\dagger\bigr]\;\in\scrD(\calH_{\text{sys}}),
\]
where $\rho_0\in\scrD(\calH_{\text{sys}})$ is the initial state on the system and $\rho_\text{env}\in\scrD(\calH_{\text{env}})$ is a given state on the environment.
Requiring \emph{Markovianity} and time-homogeneity of the evolution inevitably forces the environment to possess infinitely many degrees of freedom. In continuous time, this naturally leads to a description in terms of bosonic quantum fields.

\subsection{Bosonic Fock space}
The \emph{bosonic (or symmetric) Fock space} of a complex Hilbert space $\calK$ is defined as
\[
	\frF(\calK) := \bigoplus_{n\in\bbN_0} \calK^{\odot n},
\]
where $\calK^{\odot n}$ denotes the $n$-fold symmetric tensor product, i.e.,
\[
	f\in \calK^{\odot n}\;\;\Leftrightarrow\;\; f(\sigma(u_1,\ldots,u_n)) = f(u_1,\ldots,u_n)\quad\text{for any permutation $\sigma$},
\]
and by convention $\calK^{\odot 0} \coloneqq \bbC$. The bosonic Fock space $\frF(\calK)$ inherits the scalar product from $\calK$ defined by
\[
	\langle \oplus u^{(n)},\oplus v^{(n)}\rangle_{\frF(\calK)} = \sum_{n\in\bbN_0} \langle u^{(n)},v^{(n)}\rangle_{\calK^{\otimes n}}.
\]
We define the \emph{exponential vectors}
\[
	\sfe(u) = \bigoplus_{n\in \bbN_0} \frac{1}{\sqrt{n!}}u^{\otimes n},\qquad u\in\calK,
\]
and the distinguished vector $\Upomega \coloneqq \sfe(0) = 1\oplus 0 \oplus\cdots \in\frF(\calK)$, called the \emph{vacuum vector}.

It turns out that the set of \emph{exponential vectors} $\frE(\calK)$ is \emph{total} in $\frF(\calK)$, i.e., the linear span of $\frE(\calK)$ is dense in $\frF(\calK)$. This fact will be helpful for us in the future. Since,
\[
	\langle \sfe(u), \sfe(v)\rangle_{\frF(\calK)} = \sum_{n\in\bbN_0} \frac{1}{n!}\langle u,v\rangle_\calK^n = e^{\langle u,v\rangle_\calK},\qquad f,g\in\calK,
\]
the exponential vectors are normalizable. These normalized exponential vectors 
\[
	\uppsi(u) = e^{-\frac{1}{2}\|u\|_\calK^2}\sfe(u),\qquad u\in\calK,
\]
are commonly called \emph{coherent vectors}.

For time-continuous noise, the canonical choice is
\[
	\calK = L^2(\bbR_+,\uplambda; \mathbb{C}^d) \cong L^2(\bbR_+,\uplambda)\otimes \mathbb{C}^d,
\]
where $d$ represents the number of \emph{noise channels}. From now on, we will only consider the canonical choice and call $\frF(\calK)$ our noise environment.

%\begin{remark}
%	As a side remark, one can also consider the free Fock space $\bigoplus_{n\in\bbN_0} \calK^{\odot n}$
%\end{remark}

\subsubsection*{Field operators: Creation, annihilation, and gauge processes}
On the bosonic Fock space $\frF(\calK)$, one defines operator-valued processes:
\begin{itemize}[itemsep=0.0em]
	\item[] the \emph{annihilation} processes $\sfA_j(t)$,
	\item[] the \emph{creation} processes $\sfA_j^\dagger(t)$,
	\item[] the \emph{gauge (or number)} processes $\Uplambda_{ij}(t)\coloneqq \sfA_i^\dagger\sfA_j(t)$ for $i,j = 1,\dots,d$.
\end{itemize}
Heuristically, $\sfA_j(t)$ annihilates a quantum in channel $j$ arriving before time $t\ge 0$, $\sfA_j^\dagger(t)$ creates such a quantum, and $\Uplambda_{ij}(t)$ counts quanta between channels. Together, these processes encode absorption, emission, and counting statistics in a unified operator-theoretic framework and form the building blocks of quantum stochastic calculus, serving as driving noises in the Hudson-Parthasarathy theory of quantum processes.

\vspace{-0.8em}
\paragraph{Annihilation and creation processes.} For each channel $j=1,\dots,d$, the annihilation and creation operators are defined on exponential vectors $\sfe(u)$ by
\[
	\sfA_j(t) \sfe(u) \coloneqq \langle \mathbf{1}_{[0,t]},u_j\rangle \sfe(u), \qquad
	\sfA_j^\dagger(t)\sfe(u) \coloneqq \frac{\d}{\d\varepsilon}\Big|_{\varepsilon=0} \sfe(u + \varepsilon \mathbf{1}_{[0,t]} e_j),
\]
where $u=(u_1,\dots,u_d)\in \calK$ and $e_j$, $j=1,\ldots,d$, denotes the canonical basis of $\mathbb{C}^d$. 

These operators are densely defined, mutually adjoint, and they satisfy the \emph{canonical commutation relations}
\begin{align}\label{eq:ccr}\tag{\textsf{CCR}}
	[\sfA_i(t), \sfA_j^\dagger(s)] = \delta_{ij}\min(t,s)\,\sfI_{\frF(\calK)}, \qquad [\sfA_i(t), \sfA_j(s)] = [\sfA_i^\dagger(t), \sfA_j^\dagger(s)] = 0.
\end{align}
These relations give rise to the quantum It\^o table that we will see in the following section.

With respect to the coherent vector $\uppsi(u)\in\frF(\calK)$, $u\in\calK$, one has
\begin{gather*}
	\langle \uppsi(u), \sfA_j(t) \uppsi(u) \rangle = \langle \mathbf{1}_{[0,t]},u_j\rangle = \overline{\langle \uppsi(u),\sfA_j^\dagger(t) \uppsi(u) \rangle}, \\[0.2em]
	\langle \uppsi(u), [\sfA_i(t), \sfA_j^\dagger(s)] \uppsi(u) \rangle = \delta_{ij} \min(t,s).
\end{gather*}
In particular, in the vacuum vector $\Upomega=\uppsi(0)$, we find
\[
	\langle \Upomega, \sfA_j(t) \Upomega \rangle = \overline{\langle \Upomega,\sfA_j^\dagger(t) \Upomega \rangle} = 0,\qquad \langle \Upomega, \sfA_i(t)\sfA_j^\dagger(s) \Upomega \rangle = \delta_{ij} \min(t,s),
\]
which shows that the self-adjoint field operators $\sfB_j=\sfA_j+ \sfA_j^\dagger$ reproduce the covariance structure of classical Brownian motion, i.e.,
\[
	\langle \Upomega, \sfB_i(t)\sfB_j^\dagger(s) \Upomega \rangle = \delta_{ij} \min(t,s).
\]
Thus, classical noise is recovered as a commutative subtheory of quantum noise.

\begin{exercise}
	Use the \emph{Zassenhaus formula} and the fact that
	\[
		[\sfA_j(t),[\sfA_j(t),\sfA_j^\dagger(t)]]=[\sfA_j(t),[\sfA_j(t),\sfA_j^\dagger(t)]] = 0\qquad\text{for all $t\ge 0$},
	\]
	to show that for every $r=(r_1,\ldots,r_d)\in\bbR^d$,
	\[
		\bbE_\Upomega[e^{i \sum\nolimits_j r_j\sfB_j(t)}] \coloneqq  \langle \Upomega, e^{i \sum\nolimits_jr_j\sfB_j(t)} \Upomega \rangle = e^{-\frac{1}{2}\sum\nolimits_j r_j^2 \,t}.
	\]
	Conclude from this that the vector of field operators $(\sfB_1(t),\ldots,\sfB_d(t))$ is a $d$-dimensional Gaussian \emph{random variable} with mean $0$ and covariance $\Sigma =t\sfI_{\bbR^d}$.
%\begin{align*}
%	\bbE_\Upomega[e^{ir_j\sfB_j(t)}] &\coloneqq  \langle \Upomega, e^{ir_j\sfB_j(t)} \Upomega \rangle = \langle \Upomega, e^{ir_j(\sfA_j(t) +\sfA_j^\dagger(t))} \Upomega \rangle \\
%	&= \langle \Upomega, e^{ir_j\sfA_j^\dagger(t)}e^{ir_j\sfA_j(t)}e^{-r_j^2[\sfA_j(t), \sfA_j^\dagger(t)]/2} \Upomega \rangle \\
%	&= e^{-r_j^2\,t/2}\langle e^{-ir_j\sfA_j(t)}\Upomega, e^{ir_j\sfA_j(t)} \Upomega \rangle = e^{-r_j^2 \,t/2}
%\end{align*}
\end{exercise}


\vspace{-0.8em}
\paragraph{Gauge processes.}
The gauge processes describe the flow of quanta between channels. On exponential vectors, they act as
\[
	\Uplambda_{ij}(t)\sfe(u) = (\sfA_i^\dagger\sfA_j)(t)\sfe(u) = \langle \mathbf{1}_{[0,t]},u_i\rangle \sfA_j^\dagger\sfe(u),
\]
and may be interpreted as integrated number operators. In particular, the diagonal processes $\Uplambda_{jj}(t)$ count the number of quanta in channel $j$ up to time $t\ge 0$, i.e.,
\[
	\langle \uppsi(u), \Uplambda_{ii}(t) \uppsi(u) \rangle = |\langle \mathbf{1}_{[0,t]},u_i\rangle|^2.
\]

\begin{remark}[Adaptedness and causality]
For each $t\ge 0$, the operators $\sfA_j(t)$, $\sfA_j^\dagger(t)$ and $\Uplambda_{ij}(t)$ act nontrivially only on the tensor factor $\frF(L^2([0,t],\uplambda))$ corresponding to times up to $t\ge 0$. This adaptedness property encodes quantum causality, giving rise to a notion of filtration in this context, and ensures that increments over disjoint time intervals commute. Unfortunately, we will not have time to dive into more detail here. \hfill$\diamondsuit$
\end{remark}

\subsection{Fock-Wiener isometry}
\label{subsec:fock-wiener}

The Wiener-It\^o-Segal isomorphism provides a precise mathematical link between classical and quantum noise, which identifies the canonical bosonic Fock space with an $L^2$-space over classical Wiener space.

More precisely, let $(\calC(\bbR_+;\bbR^d), \calF, \bbW_d)$ be a classical Wiener probability space with canonical process $(W_t^1,\ldots,W_t^d)$. Then there exists a unitary isomorphism
\[
	\calU : \frF(\calK) \to L^2(\calC(\bbR_+;\bbR^d), \bbW_d),
\]
called the \emph{Fock-Wiener (or Wiener-It\^o-Segal) isometry}, with the following properties:
\begin{enumerate}[itemsep=0.0em, label=(\roman*)]
	\item The vacuum vector $\Upomega$ is mapped to the constant function $1$.
	\item Exponential vectors correspond to stochastic exponentials of Brownian motion, i.e.,
	\[
		\calU\bigl(\uppsi(u\mathbf{1}_{[0,t]})\bigr)(\omega) = \exp\biggl(\sum\nolimits_j \int_0^t u_j(s)\dd W_s^j(\omega) - \frac{1}{2}\sum\nolimits_j\|u_j\mathbf{1}_{[0,t]}\|_\calK^2\biggr)\quad\text{for $t\ge 0$}.
	\]
	Recall that the right-hand side is an exponential martingale and that the family of such functions is total in $L^2(\calC(\bbR_+;\bbR^d), \bbW_d)$.
%	\begin{align*}
%		&\bigl\langle \calU\bigl(\sfe(u\mathbf{1}_{[0,t]})\bigr),\calU\bigl(\sfe(v\mathbf{1}_{[0,t]})\bigr)\bigr\rangle \\
%		&\qquad= \bbE\left[\exp\biggl( \int_0^t \bigl(u(s)+v(s)\bigr)\dd W_s - \frac{1}{2}\int_0^t \Bigl(|u(s)|^2 + |v(s)|^2\Bigr)\dd s  \biggr)\right] \\
%		&\qquad= \exp\biggl( \int_0^t u(s)v(s)\dd s \biggr)\bbE\left[ \calU\bigl(\sfe((u+v)\mathbf{1}_{[0,t]})\bigr)(W)\right] = \langle \sfe(u\mathbf{1}_{[0,t]}),\sfe(v\mathbf{1}_{[0,t]})\rangle
%	\end{align*}

	\item Multiple Wiener integrals of order $n$ correspond to the $n$-particle sector $\calK^{\odot n}$.
\end{enumerate}

Under this isometry, the self-adjoint field operator $\sfB_j(t)$ acts as multiplication by the classical Brownian motion $W_t^j$. In particular,
\[
	\calU \sfB_j(t) \calU^{-1} = W_t^j,
\]
viewed as a multiplication operator on $L^2(\calC(\bbR_+;\bbR^d), \bbW_d)$. This identification shows that classical stochastic calculus is faithfully embedded into quantum stochastic calculus as the restriction to a commuting subalgebra of field operators.

\begin{remark}
	\begin{enumerate}[itemsep=0.0em, label=(\arabic*)]
		\item Notice that the coherent states $\uppsi(u)\in\frF(\calK)$ play the role of changing the reference measure $\bbW_d$ by the drift field $u\in \calK$.
		\item A similar construction holds for Poisson processes on the Skorokhod space.
	\end{enumerate}
	
\end{remark}


\newpage

\section{Hudson-Parthasarathy Theory}
\label{sec:HP}

Having introduced quantum noise, we now describe the dynamics of systems driven by such noise. This is accomplished by the Hudson-Parthasarathy (HP) theory of quantum stochastic differential equations (QSDE), where the bosonic Fock space is used as a model for the environment Hilbert space, i.e., $\calH_\text{env} = \frF(\calK)$.

\begin{itemize}
	\item Introduce It\^o integral in this context
\end{itemize}

\subsection{Quantum It\^o calculus}
Quantum stochastic calculus is built upon the quantum It\^o table. For the fundamental differentials, one has
\[
\begin{aligned}
	d\sfA_i(t) d\sfA_j^\dagger(t) &= \delta_{ij} \dd t, \
d\Lambda_{ij}(t), dA_k^\dagger(t) &= \delta_{jk}, dA_i^\dagger(t), \
dA_i(t), d\Lambda_{jk}(t) &= \delta_{ij}, dA_k(t),
\end{aligned}
\]
with all other quadratic differentials vanishing. These rules replace the classical quadratic variation.

\subsection{Hudson-Parthasarathy equation}
Let $H \in \mathcal{O}(\calH)$ and $L_1,\dots,L_d \in \mathcal{B}(\calH)$. The HP equation for a unitary process $U_t$ on $\calH \otimes \frF$ reads
\begin{align*}
dU_t = \Bigl( \sum_{j=1}^d L_j{\otimes} \dd \sfA_j^\dagger(t) - \sum_{j=1}^d L_j^\dagger {\otimes} \dd \sfA_j(t) - \Bigl(iH + \tfrac12 \sum_{j=1}^d L_j^\dagger L_j\Bigr) dt \Bigr) U_t,
  \quad U_0 = I.
  \end{align*}
This equation defines a unitary adapted cocycle and represents the joint evolution of system and environment.

\subsection{Stratonovich versus It\^o formulations}
\label{subsec:ito-strat-hp}

As in the classical theory, quantum stochastic differential equations admit both It\^o and Stratonovich formulations. The It\^o form is algebraically convenient and is intrinsic to the Hudson-Parthasarathy framework, while the Stratonovich form is often conceptually closer to Hamiltonian dynamics.

Let us write the HP equation in It\^o form as
\[
dU_t = G_{\text{It\^o}}(t), U_t,
\]
where
\[
G_{\text{It\^o}}(t)
:= \sum_j L_j , dA_j^\dagger(t)
* \sum_j L_j^\dagger , dA_j(t)
* \Bigl(iH + \tfrac12 \sum_j L_j^\dagger L_j\Bigr) dt.
\]

The corresponding Stratonovich form is defined by the requirement that the usual Leibniz and chain rules hold. One may formally write
\[
dU_t = G_{\mathrm{Strat}}(t) \circ U_t,
\]
where the Stratonovich generator is
\[
G_{\mathrm{Strat}}(t) := \sum_j L_j , \circ dA_j^\dagger(t)* \sum_j L_j^\dagger , \circ dA_j(t)* iH ; dt.
\]

The relation between the two forms mirrors the classical correction term: the It\^o drift contains the additional dissipative contribution $-\tfrac12 \sum_j L_j^\dagger L_j$, which arises from the non-vanishing quadratic products $dA_j(t) dA_j^\dagger(t) = dt$ in the quantum It\^o table.

In particular, the Stratonovich form makes explicit that the unitary dynamics are generated by a \emph{stochastic Hamiltonian}
\[
H_{\mathrm{eff}}(t)
= H + i \sum_j \bigl( L_j , \dot{A}_j^\dagger(t) - L_j^\dagger , \dot{A}_j(t) \bigr),
\]
while the It\^o form encodes the same dynamics together with the irreversible drift required for complete positivity.

This distinction is especially useful when comparing HP equations with the classical stochastic Schr\"odinger equations of Lecture~1, which were naturally written in Stratonovich form.

\begin{remark}[Classical versus quantum noise]
In the classical setting of Lecture~1, a unitary-valued stochastic evolution driven by Brownian motion takes the Stratonovich form
\[
dU_t = i H \circ dW_t , U_t,
\]
which corresponds to a random Hamiltonian perturbation and preserves unitarity pathwise. Passing to It\^o form introduces the familiar correction $-\tfrac12 H^2 dt$, reflecting the quadratic variation of Brownian motion.

In the quantum setting, the Stratonovich HP equation
\[
dU_t = \sum_j L_j \circ dA_j^\dagger(t) , U_t - \sum_j L_j^\dagger \circ dA_j(t) , U_t - iH dt , U_t
\]
plays an analogous role: it represents a stochastic Hamiltonian coupling between system and field. However, the non-commutativity of the quantum noise leads, in It\^o form, to the additional drift term $-\tfrac12 \sum_j L_j^\dagger L_j dt$, which has no classical analogue and is ultimately responsible for irreversible Lindblad dynamics after tracing out the environment.

Thus, classical noise produces diffusion on the unitary group, while quantum noise produces both diffusion and dissipation---a distinction that lies at the heart of open quantum system theory.
\end{remark}

\subsection{From HP equations to Lindblad dynamics}
Let $\rho_0 \in \mathcal{D}(\calH)$ and consider the vacuum state on the Fock space. The reduced system state is given by
\[
\rho_t = \mathrm{Tr}_{\frF}\bigl(U_t (\rho_0 \otimes |\Upomega\rangle\langle\Upomega|) U_t^\dagger\bigr).
\]
A direct computation using the quantum It\^o calculus yields the Lindblad equation
\[
\frac{d}{dt}\rho_t = -i[H,\rho_t] + \scrL(\rho_t),\qquad \scrL(\rho) \coloneqq - \tfrac{1}{2}\sum\nolimits_j [L_j^\dagger,[L_j,\rho]]
\]
Thus, every GKSL generator admits a unitary dilation driven by quantum noise.

\subsection{Worked example: amplitude damping}
Consider a single qubit with $L = \sqrt{\gamma} \sigma_-$ and $H = 0$. The HP equation becomes
\[
	dU_t = \bigl( \sqrt{\gamma} \sigma_-  dA^\dagger(t) - \sqrt{\gamma} \sigma_+  dA(t) - \tfrac{\gamma}{2} \sigma_+ \sigma_- dt \bigr) U_t.
\]
Tracing out the Fock space yields the master equation
\[
  \frac{\d}{\d t}\rho_t = -\tfrac{\gamma}{2}[\sigma_+,[\sigma_-,\rho_t]],
\]
which describes spontaneous emission. This dynamics cannot be obtained from any classical-noise-driven stochastic Schr\"odinger equation.

\subsection{Outlook: control and measurement}
Hudson-Parthasarathy equations provide the natural starting point for quantum filtering, continuous measurement, and feedback control. In this framework, control inputs appear as adapted processes acting on the system operators, while measurement corresponds to observing commuting output fields derived from the quantum noise.

\newpage


\subsection{Boson Fock spaces}

Consider a (complex) Hilbert space $\calH$ with scalar product $\langle\cdot,\cdot\rangle_\calH$. The symmetric Fock space associated with $\calH$ is
\[
	\frF =\frF_\mathsf{sym}(\calH) \coloneqq \bigoplus_{n\in \bbN_0} \calH^{\odot n},\qquad \calH^{\odot 0} = \bbC,
\]
where $\odot$ denotes the symmetric tensor product such that
\[
	\calH^{\odot n} = \Bigl\{ f\in\calH^{\otimes n}: f(x_{\sigma_1},\ldots,x_{\sigma_n})=f(x_1,\ldots,x_n)\;\;\text{for every permutation $\sigma$}\Bigr\}.
\]

 The Fock space $\frF$ inherits the scalar product from $\calH$ defined by
\[
	\langle \oplus f^{(n)},\oplus g^{(n)}\rangle_\frF = \sum_{n\in\bbN_0} \langle f^{(n)},g^{(n)}\rangle_{\calH^{\otimes n}}.
\]
We define the \emph{vacuum vector} $\Upomega=1\oplus 0 \oplus 0^{\otimes 2}\oplus\cdots \in\frF$, and the \emph{exponential vectors}
\[
	\sfe(f) = \bigoplus_{n\in \bbN_0} \frac{1}{\sqrt{n!}}f^{\otimes n},\qquad f\in\calH.
\]
It turns out that the family of exponential vectors $\frE$ is \emph{total} in $\frF$, i.e., the linear span of $\frE$ is dense in $\frF$. This fact will be helpful for us in the future. Since,
\[
	\langle \sfe(f), \sfe(g)\rangle_{\frF} = \sum_{n\in\bbN_0} \frac{1}{n!}\langle f,g\rangle_\calH^n = e^{\langle f,g\rangle_\calH},\qquad f,g\in\calH,
\]
the exponential vectors are normalizable. These normalized exponential vectors 
\[
	\uppsi(f) = e^{-\frac{1}{2}\|f\|_\calH^2}\sfe(f),\qquad f\in\calH,
\]
are commonly known as \emph{coherent vectors}.

\subsubsection{Weyl and field operators}

For any $f\in\calH$, we define the \emph{Weyl operator} on exponential vectors by
\[
	\sfW(f)\sfe(g) := \exp \left(-\langle f,g\rangle_\calH - \frac{1}{2}\|f\|_\calH^2\right)\sfe(f+g),\qquad g\in\calH.
\]
Weyl operators play an essential role in the setup of Fock spaces. For one, they generate coherent states by acting on the vacuum state, i.e.,
\[
	\sfW(f)\Upomega = e^{-\frac{1}{2}\|f\|_\calH^2}\sfe(f) = \uppsi(f),\qquad f\in\calH.
\]
Moreover, they give the means to map any element $f\in\calH$ to unitary operators on $\frF$ that satisfy the \emph{canonical commutation relation} (CCR):

\begin{proposition}
	The Weyl operator $\sfW(f)$ is a unitary operator and satisfies 
	\begin{enumerate}[label=(\roman*)]
		\item $\sfW^\dagger(f) = \sfW(-f)$.
		\item $\sfW(f)\sfW(g) = e^{-i\mathrm{Im}(\langle f,g\rangle_\calH)}\sfW(f+g) = e^{-2i\mathrm{Im}(\langle f,g\rangle_\calH)}\sfW(g)\sfW(f).$
	\end{enumerate}
	Property (ii) is the Weyl form of the canonical commutation relation (CCR).
\end{proposition}
\begin{proof}
	For any $g\in \calH$,
	\begin{align*}
	\langle \sfW(f)\sfe(g), \sfW(f)\sfe(g)\rangle_{\frF} &= \exp \left(-2\langle f,g\rangle_\calH - \|f\|_\calH^2\right)\langle \sfe(f+g),\sfe(f+g)\rangle_\frF \\
	&= \exp \left(-2\langle f,g\rangle_\calH - \|f\|_\calH^2 + \|f+g\|_\calH^2\right) \\
	&= e^{\|g\|_\calH^2} = \langle \sfe(g),\sfe(g)\rangle_{\frF}.
\end{align*}
Hence, $\sfW(f)$ preserves inner products on $\frE$. Since $\frE$ is dense in $\frF$, $\sfW(f)$ extends uniquely to an isometry on $\frF$. 

In a similar fashion, we compute
\begin{align*}
	\sfW(-f)\sfW(f)\sfe(g) = \sfe(g) = \sfW(f)\sfW(-f)\sfe(g)\qquad\forall g\in\calH,
\end{align*}
i.e., $\sfW(-f)\sfW(f)$ is the identity on the dense set $\frE$. In particular, $\sfW(f)$ is surjective and an isometry, i.e., $\sfW(f)$ is unitary with $\sfW^\dagger(f) = \sfW(-f)$.

As for the last property, we observe that
\begin{align*}
	\sfW(f)\sfW(g)W(-(f+g))\sfe(h) &= e^{\langle f+g,h\rangle_\calH - \frac{1}{2}\|f+g\|_\calH^2}\sfW(f)\sfW(g)\sfe(-(f+g)+h) \\
	&= e^{\langle f+g,h\rangle_\calH - \frac{1}{2}\|f\|_\calH^2-\langle g,-f+h\rangle_\calH - \mathrm{Re}(\langle g,f\rangle_\calH)}\sfW(f)\sfe(-f+h)\\
	&= e^{-i\mathrm{Im}(\langle f,g\rangle_\calH)}\sfe(h),
\end{align*}
and hence, $\sfW(f)\sfW(g)\sfW(-(f+g)) = e^{-i\mathrm{Im}(\langle f,g\rangle_\calH)}I_\frF$. We then conclude by using property \emph{(i)} of Weyl operators.
\end{proof}

Since $\sfW(f)$ is unitary for every $f\in\calH$, the family $\{\sfW(tf)\}_{t\in\bbR}$ forms a one-parameter (strongly continuous) group of unitaries. In particular, due to Stone's theorem, it has a corresponding Hermitian operator $\sfP(f)$ such that
\[
	\sfW(tf) = \exp(it \sfP(f)).
\]
We further define the following operators
\[
	\sfQ(f):=-\sfP(if),\qquad \sfA^-(f):= \frac{\sfQ(f)+i\sfP(f)}{2},\qquad \sfA^+(f):= \frac{\sfQ(f)-i\sfP(f)}{2}.
\]
The operators $\sfA^\pm$ are called the \emph{field operators} and will play an essential role as they form the creation/annihilation operators on Fock spaces.


\begin{proposition}
	The following are true: For any $f,g\in\calH$,
	\begin{enumerate}[label=(\roman*)]
		\item $\frE$ is a core for $\sfP(f)$ and $[\sfP(f),\sfP(g)] = 2i \mathrm{Im}(\langle f,g\rangle_\calH)\sfI_\frF$.
		\item $\sfA^-(f)\sfe(g) = \langle f,g\rangle_\calH\,\sfe(g)$,\quad $\sfA^+(f)\sfe(g) = \frac{d}{dt}\sfe(g+tf)|_{t=0}$.
		\item $\sfW^\dagger(f)\sfA^-(g)\sfW(f) = \sfA^-(g) + \langle g,f\rangle_\calH\sfI_\frF$,\quad $\sfW^\dagger(f)\sfA^+(g)\sfW(f) = \sfA^+(g) + \langle f,g\rangle_\calH\sfI_\frF$.
		\item $[\sfA^-(f),\sfA^-(g)] = [\sfA^+(f),\sfA^+(g)] = 0$,\quad $[\sfA^-(f),\sfA^+(g)]=\langle f,g\rangle_\calH \sfI_\frF$,
	\end{enumerate}
	i.e., the field operators $\sfA^\pm$ satisfy the canonical commutation relation.
\end{proposition}

\begin{remark}
	On the finite particle vectors, the field operators act as
	\[
		\sfA^-(f)g^{\otimes n} = \sqrt{n}\langle f,g\rangle_\calH g^{\otimes (n-1)},\qquad \sfA^+(f)g^{\otimes (n-1)} = \frac{1}{\sqrt{n}}\sum_{k=0}^{n-1} g^{\otimes k}\otimes f\otimes g^{\otimes (n-1-k)}.
	\]
%	It is common in Physics to formally write $\sfA_x^\pm=\sfA^\pm(\delta_x)$ when $\calH=L_\bbC^2(\bbR^d)$. Since $\delta_x\notin \calH$, this is obviously not meaningful. However, since the set of continuous $\calC_c(\bbR^d;\bbC)$ is dense in $L_\bbC^2(\bbR^d)$, the family $\{\sfe(g):g\in \calC_c(\bbR^d;\bbC)\}$ it total in $\frF$. In particular, the expression
%	\[
%		\begin{aligned}
%			\sfA_x^- g^{\otimes n} &= \sqrt{n}g(x)g^{\otimes(n-1)} \\
%			\sfA_x^+g^{\otimes n}
%		\end{aligned}
%		,\qquad g\in \calC_c(\bbR^d;\bbC)
%	\]
%	is meaningful.
\end{remark}

%\begin{remark}
%	Notice that if $\{\psi_j\}_{j}$ forms an orthonormal basis for $\calH$, then the family $\{\sfA^\pm_i:=\sfA^\pm(\psi_j)\}_{j}$ of field operators is a family of commuting operators and may be seen as a `basis' for the set of field operators since for any $f\in\calH$ with $f=\sum_{j} f_j\psi_j$, we have the representation
%	\begin{align*}
%		\sfA^-(f)g^{\otimes n} &= \sum\nolimits_{j} \sqrt{n}f_j\langle \psi_j,g\rangle_\calH\,g^{\otimes(n-1)} = \sum\nolimits_{j} f_j \sfA_j^-g^{\otimes n} \\
%		\sfA^+(f)g^{\otimes n} &= \sum\nolimits_j f_j\frac{1}{\sqrt{n-1}}\sum_{k\le n} g^{\otimes k}\otimes \psi_j\otimes g^{\otimes(n-k)} = \sum\nolimits_j f_j\sfA_j^+ g^{\otimes n}
%	\end{align*}
%\end{remark}

\begin{example}
	Let $\calH=L_\bbC^2([-\pi,\pi])$ be the space of square-integrable functions. Then the countable set $\{\psi_\ell(x)=e^{i\ell x}:\ell\in\bbZ\}$ forms an orthonormal basis for $\calH$. 
	\[
		\uppsi(x) = \sum\nolimits_\ell e^{i\ell x} \sfA_\ell^-
	\]
\end{example}


%From the Baker-Campbell-Hausdorf formula, one obtains
%\[
%	W(\sqrt{t}f)W(\sqrt{t}g)W(-\sqrt{t}(f+g)) = \exp\left(- \frac{t}{2}[B(f),B(g)] + O(t^{3/2})\right)
%\]
%\begin{align*}
%	-\frac{1}{2}[B(f),B(g)]\uppsi &= \lim_{t\to 0}\frac{1}{t^2}\Bigl(W(tf)W(tg)W(-t(f+g))\uppsi -\uppsi\Bigr) \\
%	&= -i\mathrm{Im}(\langle f,g\rangle_\calH)\uppsi
%\end{align*}
%\[
%	[B(f),B(g)] = 2i \mathrm{Im}(\langle f,g\rangle_\calH)
%\]
%
%\[
%	Z = itB(f+g) - \frac{t^2}{2}[B(f),B(g)] - i\frac{t^3}{12}[B(f),[B(f),B(g)]]
%\]
%\[
%	X = - \frac{t^2}{2}[B(f),B(g)] + O(t^3)
%\]
%
%\[
%	a^-(f) := \frac{B(f)+B(if)}{2},\qquad a^-(f) := \frac{B(f)-B(if)}{2i}
%\]
%\[
%	[a^-(f),a^+(f)] = [\frac{B(f)+B(if)}{2},\frac{B(f)-B(if)}{2}] = -\frac{1}{2i}[B(f),B(if)] = -\|f\|_\calH^2
%\]


\subsubsection{Second quantization}

The term \emph{second quantization} is associated with the action of lifting operators on an \emph{$k$-particle} space $\calH^{\otimes k}$ to an associated operator on the Fock space. 

We begin our discussion with the \emph{$1$-particle} case. For any bounded operator $A\in \calB(\calH)$ one defines the map
\[
	\Upgamma(A) := I\oplus \bigoplus_{n\in\bbN} A^{\otimes n},
\]
which acts on $\calH^{\otimes n}$ by
\[
	\Upgamma(A) g_1\otimes\cdots\otimes g_n = Ag_1\otimes \cdots \cdots\otimes Ag_n.
\]
Clearly, if $A$ is unitary, then so is $\Upgamma(A)$. Indeed, in this case, one has
\[
	\langle \Upgamma(A)e(g),\Upgamma(A)e(g)\rangle_{\frF(\calH)} = \sum_{n\in\bbN_0}\frac{1}{n!} \langle Ag,Ag\rangle_\calH^n = \sum_{n\in\bbN_0}\frac{1}{n!} \|g\|_\calH^n = \langle e(g),e(g)\rangle_{\frF(\calH)}.
\]

Now let $H$ be a self-adjoint operator on $\calH$ and $U_t:=\exp(itH)$ be its unitary evolution. Then, $\Upgamma(U_t)$ is a one-parameter group of unitary operators on $\frF(\calH)$. Stone's theorem then provides the existence of a densely defined Hermitian operator $\sfd\Upgamma(H)$ such that
\[
	\Upgamma(U_t) = \exp(it\,\sfd\Upgamma(H)).
\]
The generator $\sfd\Upgamma(H)$ is called the \emph{second quantization of $H$}, and takes the explicit form
\[
	\sfd\Upgamma(H) g^{(n)} = \sum_{j=1}^n g_1\otimes\cdots\otimes g_{j-1} \otimes Hg_j\otimes g_{j+1}\otimes\cdots\otimes g_n =\sum_{j=1}^n H_j g^{(n)},
\]
for any $g^{(n)}=g_1\otimes\cdots\otimes g_n\in \calH^{\otimes n}$ with $g_j\in D(H)$ and 
\[
	H_j = I_\calH\otimes\cdots\otimes I_\calH\otimes H\otimes I_\calH\otimes\cdots\otimes I_\calH,
\]
where $H$ acts on the $j$-th tensor product. The special case $H=I_\calH$ yields
\[
	\sfN g^{(n)}:=\sfd\Upgamma(I_\calH)g^{(n)} = ng^{(n)},\qquad n\in\bbN,
\]
and is called the \emph{number operator} due to its diagonal nature, with eigenvalues representing the number of particles in each configuration. Its domain is given by
\[
	D(\sfN) = \Bigl\{ \{f^{(n)}\}_{n\in\bbN_0}:\sum_{n\in\bbN_0} n^2\|f^{(n)}\|_{\calH^{\otimes n}}^2<+\infty\Bigr\}.
\]
\[
	\sfA^-_j\sfA^+_j \psi_j^{\otimes n} = \sqrt{n}\,\sfA_j^-\psi_j^{\otimes(n-1)} = \sum_{k=0}^{n-1} \psi_j^{\otimes k}\otimes \psi_j\otimes \psi_j^{\otimes(n-1-k)} = n\psi_j^{\otimes n}
\]
\[
	\sfA_j^+\sfA_j^- \sfe(\psi_k) = \delta_{jk}\sfN_j\,\sfe(\psi_j)
\]


This construction can be performed similarly for the general $k$-particle case. 

\[
	\sfd\Upgamma(H^{(k)})f^{(n)} = \sum_{j_1\ne \cdots\ne j_k} H_{j_1\cdots j_k}f^{(n)},
\]
where $H_{j_1,\ldots,j_k}$ denotes the operator where $H^{(k)}$ acts on the $(j_1,\ldots,j_k)$-th  tensor product.




\begin{example}
	Consider the Hilbert space $\calH=L_\bbC^2(\bbT)$ and the Hamiltonians,
	\[
		H^{(1)} = -\Delta \in \calO(\calH),\qquad H^{(2)} = \sfM_W\in\calO(\calH^{\otimes 2}),
	\]
	where $W:\bbT\times\bbT\to\bbR$ is an interaction potential, and $\sfM_f$ denotes the multiplication operator corresponding to $f$.
	
	Then, their second quantization is given by
	\[
		\sfd\Upgamma(H^{(1)}) = \sum_{j=1}^n H_j^{(1)},\qquad \sfd\Upgamma(H^{(2)}) = \sum_{j\ne \ell} H_{j\ell}^{(2)}\qquad\text{on $\calH^{\otimes n}$}.
	\]
	\begin{align*}
		\sfd\Upgamma(H^{(1)})\psi_j^{\otimes n} = n j^2 \psi_j^{\otimes n} = j^2 \sfA_j^+\sfA_j^-\psi_j^{\otimes n} = j^2 \sfA_j^+\sfA_j^-\psi_j^{\otimes n}
	\end{align*}
\end{example}
\[
	\psi^{(n)} = \psi_{k_1}\otimes\cdots\otimes\psi_{k_n}
\]
\begin{align*}
		\sfd\Upgamma(H^{(1)})\psi^{(n)} = \sum_{j=1}^n k_j^2\, \psi^{(n)}
	\end{align*}

\newpage


\subsubsection{Free field operators}

For each $f\in \calH$, we consider the pair $\{a^-(f),a^+(f)\}$ of operators on $\frF_\mathsf{sym}(\calH)$ defined by
\begin{align*}
	a^-(f)\Uppsi = 
\end{align*}



 satisfying the commutation relations: For any $f,g\in \calH$,
\[
	[a^\pm(f),a^\pm(g)] = 0,\qquad [a^-(f),a^+(g)] = \langle f,g\rangle,
\]
where $f\mapsto a^-(f)$ is conjugate linear and $f\mapsto a^+(f)$ is linear. Moreover, if $\Upomega\in \frF_\mathsf{sym}(\calH)$ is the \emph{vacuum vector}, then $a^-(f)\Upomega=0$ for every $f\in\calH$. The \emph{field operators} $a^-$ and $a^+$ are called \emph{creation} and \emph{annihilation} operators, respectively. On appropriate domains, the field operators are adjoints of one another, i.e., $(a^-(f))^\dagger = a^+(f)$ for every $f\in \calH$.

It is common in the physics literature to consider operator-valued distributions $\{a_x^-,a_x^+\}$ instead, where if $\calH=L_\bbC^2(\bbR^d)$, then $x\in \bbR^d$ and
\[
	a^-(f) = \int_{\bbR^d} \overline{f(x)}\,a_x\dd x,\qquad a^+(f) = \int_{\bbR^d} f(x)\,a_x\dd x.
\]
The commutation relations then simply read
\[
	[a_x^\pm, a_y^\pm] = 0,\qquad [a_x^-,a_y^+] = \delta(x-y).
\]
The \emph{number operator} is formally defined by $N_x = a_x^+a_x$, $x\in\bbR^d$. 

On the rigorous not, if $\calH$ is separable with orthonormal basis $\{\psi_i\}$, then one obtains a family of field operators $\{a_i^-,a_i^+\}$ with $a_i^\pm : a^\pm(\psi_i)$. 


\subsubsection{Gaussian states}

\begin{definition}[Gaussian states]
	A state $\Uppsi$ on $\frF_\mathsf{sym}(\calH)$ is said to be a mean-zero \emph{Gaussian} (or \emph{quasi-free}) state if it can be uniquely determined from the field operators $\{a,a^\dagger\}$ by its covariance
	\begin{align*}
		\Sigma_\Uppsi(f,g)\coloneqq \begin{pmatrix}
			\Uppsi(a^+(f)\,a^-(g)) & \Uppsi(a^-(f)\,a^-(g)) \\
			\Uppsi(a^+(f)\,a^+(g)) & \Uppsi(a^-(f)\,a^+(g))
		\end{pmatrix},\qquad f,g\in\calH.
	\end{align*}
	If the off-diagonal elements of the covariance are zero, the state $\Uppsi$ is called \emph{gauge-invariant} since it is invariant under the so-called gauge transformations of the first kind, i.e.,
	\[
		a^\pm(f) = e^{\pm i\alpha}a^\pm(f)\qquad \text{for any $\alpha\in\bbR$.}
	\]
	If the off-diagonal elements of the covariance are nonzero, the state $\Uppsi$ is called \emph{squeezed}.
\end{definition}

%Notice that a gauge-invariant mean-zero Gaussian state $\Uppsi$ on $\frF_\mathsf{sym}(\calH)$ with $\calH=L^2(\bbR^d)$ necessarily takes the form
%\begin{align*}
%	\Sigma_\Uppsi(x,y)= \begin{pmatrix}
%		0 & 0 \\
%		0 & \Uppsi(a_x^-a_y^+)
%	\end{pmatrix},\qquad x,y\in\bbR^d,
%\end{align*}
%since $a_x^-\,\Upomega = 0$ for every $x\in\bbR^d$.

\subsubsection{States invariant under free evolutions}

Consider a Hamiltonian $H$ on $\frF_\mathsf{sym}(\calH)$ with $\calH = L_\bbC^2(\bbR^d)$ and its associated 1-parameter automorphism group
\[
	\fru_t(a)=e^{itH}ae^{-itH},\qquad t\in\bbR.
\]

\begin{definition}
	The Hamiltonian $H$ on $\frF_\mathsf{sym}(\calH)$ is called \emph{free} if there exists a real-valued function $\omega\colon\bbR^d\to \bbR$ such that
	\[
		\fru_t(a_x^-) = e^{-it\omega(x)} a_x^-,\qquad x\in\bbR^d.
	\]
	In this case, the function $\omega$ is called the \emph{free 1-particle Hamiltonian}, and $H$ is said to be the \emph{second quantization} of $\omega$. Accordingly, $\fru_t$ is called a \emph{free evolution}.
\end{definition}

\begin{example}
	Consider a Hamiltonian on $\calH=L_\bbC^2(\bbR)$ and its eigensystem $\{(\lambda_i,\psi_i)\}_i$ such that $\{\psi_i\}_i$ forms an orthonormal basis for $\calH$. Setting $a_i^\pm = a^\pm(\psi_i)$, we see that
	\[
		[a_i^\pm,a_j^\pm] = 0,\qquad [a_i^-,a_j^+] = \delta_{ij},
	\]
	i.e., $a_i^\pm$ define field operators on the symmetric Fock space $\frF_\mathsf{sym}(\calK)$ with $\calK=\ell_\bbC^2$
	
	
	
	
	
	
	Defining the field operators
	\[
		a^-= \frac{1}{\sqrt{2}}\bigl( Q + iP\bigr),\qquad a^+ = \frac{1}{\sqrt{2}}\bigl(Q - iP\bigr),\qquad N:=a^+a^-
	\]
	such that $[a^-,a^+]=I$, we find that
	\[
		H = a^+a^- + \frac{1}{2} = N + \frac{1}{2}.
	\]
	Clearly, the eigenvectors of $N$ and $H$ coincide
\end{example}

\begin{definition}
	A Gaussian state $\Uppsi$ on $\frF_\mathsf{sym}(\calH)$ with $\calH=L^2(\bbR^d)$ is said to be invariant under a free evolution $\fru_t$ if
	\[
		\Uppsi(\fru_t(a_x^-)\,\fru_{s}(a_y^+)) = \Uppsi(\fru_{t-s}(a_x^-)\,a_y^+) = \Uppsi(a_x^-\,\fru_{s-t}(a_y^+))\qquad\text{for all $t\in\bbR$, $x\in\bbR^d$}.
	\]
	A \emph{Gaussian free state} is a Gaussian state that is invariant under \emph{all} free evolutions.
\end{definition}

\begin{theorem}
	A Gaussian state $\Uppsi$ on $\frF_\mathsf{sym}(\calH)$ is a Gaussian free state if and only if it is gauge-invariant and its diagonal correlations are supported on the diagonal, i.e.,
	\[
		\Uppsi(a_x^-\,a_y^+) = m(x)\delta(x-y),\qquad \Uppsi(a_x^+\,a_y^-) = n(x)\delta(x-y).
	\]
\end{theorem}

\begin{theorem}
	The field operators $\{a_x^+,a_x^-\}$ satisfying the commutation relations
	\[
		[a_x^\pm, a_y^\pm] = 0,\qquad [a_x^-,a_y^+] = m(x)\delta(x-y).
	\]
	are mean-zero Gaussian random variables w.r.t.\ the Fock vacuum state $\Uppsi_\Upomega=\langle \Upomega,\cdot\,\Upomega\rangle$, where $\Upomega\in \frF_\mathsf{sym}(\calH)$ is the Fock vacuum vector, with covariance
	\begin{align*}
		\Sigma_\Uppsi(x,y) = \begin{pmatrix}
			0 & 0 \\
			0 & m(x)
		\end{pmatrix}\delta(x-y),\qquad x,y\in\bbR^d.
	\end{align*}
Conversely, if $\{a_x^+,a_x^-\}$ are random variables with these properties, then they satisfy the commutation relations above.
\end{theorem}

\subsubsection{Boson Fock white noise}

\begin{definition}\label{def:boson-white-noise}
	A boson Fock white noise on $\bbR^d$ is a boson Fock field $\{b_{t,x}^+,b_{t,x}^-\}$ on $\bbR^{d+1}$ with vacuum vector $\Upomega$ satisfying the commutation relations
	\[
		[b_{t,x}^\pm, b_{s,y}^\pm] = 0,\qquad [b_{t,x}^-,b_{s,y}^+] = \delta(t-s)m(x)\delta(x-y),\qquad b_{t,x}^-\Upomega = 0.
	\]
\end{definition}

\begin{theorem}\label{thm:stochastic-limit}
 Let $\{a_x^+,a_x^-\}$ be Gaussian free fields w.r.t.\ the Fock vacuum state $\Uppsi_\Upomega$ with
 \[
 	\fru_t(a_x^-) = e^{-it\omega(x)}a_x^-.
 \]
 Then the rescaled field operators
 \[
 	b_{t,x}^{\lambda,\pm}\coloneqq \frac{1}{\lambda} \fru_{t/\lambda^2} (a_x^\pm)
 \]
 converges in the sense of correlator distributions to a boson Fock white noise, i.e.,
 \[
 	\lim_{\lambda\to 0} \Uppsi(b_{t,x}^{\lambda,\varepsilon_1}b_{t,y}^{\lambda,\varepsilon_2}) = \Uppsi(b_{t,x}^{\varepsilon_1}b_{t,y}^{\varepsilon_2})\qquad \varepsilon_1,\varepsilon_2\in\{+,-\},
 \]
 where $b_{t,x}^\pm$ as defined in Definition~\ref{def:boson-white-noise} with $m=2\pi \delta(\omega)$.
\end{theorem}
\begin{proof}
	Using the invariance of $\Uppsi_\Upomega$ under free evolutions, we find
	\begin{align*}
	\Uppsi_\Upomega(b_{t,x}^{\lambda,-}b_{s,y}^{\lambda,+}) &= \frac{1}{\lambda^2} \Uppsi_\Upomega(\fru_{(t-s)/\lambda^2} (a_x^-)\,a_y^+) \\
	&= \frac{1}{\lambda^2}e^{-i\omega(x)(t-s)/\lambda^2}\,\Uppsi_\Upomega(a_x^-\,a_y^+) = \frac{1}{\lambda^2}e^{-i\omega(x)(t-s)/\lambda^2}\delta(x-y).
\end{align*}
Passing to the limit $\lambda\to 0$ recovers the desired limit. All the other terms vanish.
\end{proof}


\begin{remark} Let $\calK\subset L^2(\bbR^d)$ be a set of functions for which
\[
	\int_\bbR |\langle f,e^{it\omega}g\rangle|\dd t <+\infty\qquad\forall f,g\in \calK.
\]
Since $t\mapsto \langle f,e^{it\omega}f\rangle$ is positive definite for each $f\in\calK$, Bochner's theorem implies that the sesquilinear form
\[
	\langle f,2\pi\delta(\omega)g\rangle \coloneqq \int_\bbR \langle f,e^{it\omega} 
	g\rangle\dd t,
\]
is a pre-scalar product. With $(\cdot|\cdot)$, the set $\calK$ becomes a pre-Hilbert space, which can be completed to obtain a Hilbert space, still denoted by $\calK$. The function $m=2\pi \delta(\omega)$ has to be understood in this sense, and only makes sense for functions in $\calK$.
\end{remark}


The operators
\[
	B_t^-(f)\coloneqq \int_0^t\!\!\int_{\bbR^d}\overline{f(x)} \,b_{s,x}^-\dd x\dd s,\qquad B_t^+(f)\coloneqq \int_0^t\!\! \int_{\bbR^d}f(x) \,b_{s,x}^+\dd x\dd s,
\]
define \emph{quantum Brownian motions}. The self-adjoint \emph{(momentum)} operators
\[
	P_t(f)\coloneqq \frac{1}{i}\bigl[ B_t^-(f) - B_t^+(f)\bigr],
\]
form a commuting family of classical random variables whose statistics in the Fock vacuum state $\Uppsi_\Upomega=\langle \Upomega,\cdot\,\Upomega\rangle$ is completely determined by the relation
\[
	\Uppsi(e^{iP_t(f)}) = \exp\left(-\frac{t}{2}\|f\|_{L^2(\bbR^d)}^2\right).\qquad\text{\alert{check!}}
\]


\subsubsection{Gaussian equilibrium states: The KMS condition}
For any states $a,b\in\calA$, the map $t\mapsto \Uppsi(a\,\fru_t(b))$ can be analytically continued and satisfies the so-called \emph{KMS condition} at inverse temperature $\beta>0$:
\[
	\Uppsi(a\,\fru_{t+i\beta}(b)) = \Uppsi(\fru_t(a)\,b)\qquad\forall\, a,b\in \calA.\qquad\text{\alert{check!}}
\]


\subsection{Composite systems}

\begin{definition}
	A composite system of two given quantum dynamical  systems $S=\{\calH_S,H_S\}$, $R=\{\calH_R,H_R\}$ is a quantum dynamical system of the form
	\[
		\{\calH_S\otimes\calH_R,H_{SR}\},\qquad H_{SR} = H_S\otimes 1_R + 1_S\otimes H_R + H_I,
	\]
	where $H_I$ is called the \emph{interaction Hamiltonian} and contains all the new physics of the composite system, while $H_0\coloneqq H_S\otimes 1_R + 1_S\otimes H_R$ is called the free Hamiltonian.
\end{definition}

We will consider \emph{scaled} total Hamiltonians
\[
	H^\lambda \coloneqq H_0 + \lambda H_I,
\]
and the following unitary evolutions:
\begin{gather*}
	\textit{free evolution}\quad V_t^0 = e^{-it H_0},\qquad\textit{total evolution}\quad V_t^\lambda = e^{-it H^\lambda},\\
	\textit{interacting representation evolution}\quad U_t^\lambda = (V_t^0)^\dagger V_t^\lambda,
\end{gather*}
where $U_t^\lambda$ satisfies the Schr\"odinger equation in the interaction picture:
\[
	\partial_t U_t^\lambda = -i\lambda H_I(t) U_t^\lambda,\qquad U_0^\lambda = I,
\]
with the time dependent Hamiltonian $H_I(t) = (V_t^0)^\dagger H_I V_t^0$.

For simplicity, we will make the following assumptions on system $S=\{\calH_S,H_S\}$, the reservoir $R=\{\calH_R,H_R\}$, and the interaction Hamiltonian $H_I$.

\subsubsection{The reservoir}

The reservoir $R=\{\calH_R,H_R\}$ is given by the Hilbert space $\frF_\mathsf{sym}(\calH)$ with $\calH=L_\bbC^2(\bbR^d)$, a free Hamiltonian $H_R$ with continuous spectrum $\bbR$, and a mean-zero Gaussian free state $\Uppsi_R$ such that
\[
	\int_\bbR |\Uppsi_R(a_x^{\varepsilon_1}\,\fru_t(a_y^{\varepsilon_2}))|\dd t <+\infty,\qquad \varepsilon_1,\varepsilon_2\in\{+,-\}.
\]
In particular, $\Uppsi_R$ is characterized by the covariances
\[
	\Uppsi_R(a_x^-a_y^+) = m(x)\delta(x-y),\qquad \Uppsi_R(a_x^+a_y^-) = n(x)\delta(x-y).
\]
Since $H_R$ is free, there exists a function $\omega$, for which
\[
	\fru_t(a_x^-) = e^{itH_R}a_x^-e^{-itH_R} = e^{-it\omega(x)}a_x^-,
\]
where $\omega$ describes the 1-particle evolution.

An example of a free reservoir Hamiltonian is given by
\[
	H_R = \int_{\bbR^d} \omega(x)\,a_x^+a_x^-\dd x,
\]
where $\omega$ is a smooth cutoff function.

\subsubsection{The system Hamiltonian}

For simplicity, we shall assume that the system Hamiltonian $H_S$ has a discrete spectrum such that
\[
	H_S = \sum_j \lambda_j P_j,
\]
where $\lambda_j$ are the eigenvalues and $P_j$ are their corresponding spectral projections.

\subsubsection{The interaction Hamiltonian}

We consider dipole-type interaction Hamiltonians of the form
\[
	H_I = \int_{\bbR^d} \bigl[D(x)\otimes a_x^+ + D^\dagger(x)\otimes a_x^-\bigr]\dd x,
\]
where $\{D(x):x\in\bbR^d\}$ is a family of system operators called the \emph{response terms}.

With the spectral projections of $H_S$, we may express $H_I$ as
\[
	H_I = \sum_{j,\,k}\int_{\bbR^s}  \bigl[P_j D(x)P_k\otimes a_x^+ + P_k D^\dagger(x)P_j\otimes a_x^-\bigr]\dd x,
\]
and hence, the time-dependent Hamiltonian reads
\begin{align*}
	H_I(t) &= \sum_{j,\,k}\int_{\bbR^d}  \bigl[P_j D(x)P_k\otimes e^{it(\omega(x)+\lambda_j-\lambda_k)} a_x^+ + P_k D^\dagger(x)P_j\otimes e^{-it(\omega(x)+\lambda_j-\lambda_k)} a_x^-\bigr]\dd x \\
	&= \sum_q\sum_{\lambda_k-\lambda_j\, =\, \eta_q} \int_{\bbR^d}  \bigl[P_j D(x)P_k\otimes e^{it(\omega(x)-\eta_q)} a_x^+ + P_k D^\dagger(x)P_j\otimes e^{-it(\omega(x)-\eta_q)} a_x^-\bigr]\dd x \\
	&= \sum_q\int_{\bbR^d} \bigl[D_q(x)\otimes e^{it(\omega(x)-\eta_q)} a_x^+ + D_q^\dagger(x)\otimes e^{-it(\omega(x)-\eta_q)} a_x^-\bigr]\dd x,
\end{align*}
where the system operators
\[
	D_q(x) \coloneqq \sum_{\lambda_k-\lambda_j\, =\, \eta_q} P_j D(x)P_k\qquad\text{satisfy}\quad e^{it H_S}D_q(x)e^{-it H_S} = e^{-it \eta_q}D_q(x).
\]
To simplify things drastically, we assume that $q=1$ and $D_1(x) = \chi(x)D$ for some smooth cutoff function $\chi$ and a fixed system operator $D$. In this case, we obtain
\begin{align*}
	H_I(t) &= \int_{\bbR^d} \bigl[D\otimes \chi(x) e^{it(\omega(x)-\eta)} a_x^+ + D^\dagger\otimes \overline{\chi(x)}e^{-it(\omega(x)-\eta)} a_x^-\bigr]\dd x.
\end{align*}

\subsection{The weak interaction stochastic limit}

Altogether, we arrive at the rescaled Schr\"odinger equation in the interaction picture
\begin{align}\label{eq:rescaled-schrodinger-interaction}
	U_{t/\lambda^2}^\lambda = I- i \int_0^t H_I^\lambda(s)\, U_{s/\lambda^2}^\lambda\dd s,
\end{align}
with $H_I^\lambda(t) =  D\otimes b_t^{\lambda,+} + D^\dagger\otimes b_t^{\lambda,-}$, where, due to Theorem~\ref{thm:stochastic-limit},
\[
	b_t^{\lambda,\pm} = \frac{1}{\lambda}a^\pm(\chi e^{i(t/\lambda^2)(\omega-\eta)})\longrightarrow b_t^\pm\qquad\text{in the sense of correlators},
\]
and therefore,
\[
	H_I^\lambda(t)\longrightarrow H_t = D\otimes b_t^+ + D^\dagger\otimes b_t^-,\qquad U_{t/\lambda^2}^\lambda\longrightarrow U_t,
\]
where $U_t$ satisfies the formal stochastic differential equation
\begin{align}\label{eq:sde}
	\d U_t = -i(D\otimes \d B_t^+ + D^\dagger\otimes \d B_t^-)U_t,\qquad B_t^\pm = \int_0^t b_s^\pm \dd s.
\end{align}
Under certain assumptions on $\omega$, this SDE has a unique solution.

The SDE \eqref{eq:sde} in natural-time order (or It\^o form) is the SDE given by
\begin{align}\label{eq:sde-normal}
	\d U_t = -i(D\,\d B_t^+U_t + D^\dagger U_t\, \d B_t^-) - \gamma_-D^\dagger DU_t\dd t,
\end{align} 
obtained by commuting $\dd B_t^- U_t$ and using the fact that the solution $U_t$ to \eqref{eq:sde} satisfies
\begin{align*}
	\begin{aligned}
	[b_t^-,U_t] &= -i\gamma_- DU_t,\\
	[U_t^\dagger,b_t^+] &= i\overline{\gamma}_- U_t^\dagger D^\dagger,\\
	[b_t^-,U_t^\dagger] &= i\gamma_-U_t^\dagger D,
	\end{aligned}\qquad \text{where}\quad\gamma_- \coloneqq \int_{-\infty}^0 \langle \chi,e^{-it(\omega-\eta)}\chi\rangle\dd t.
\end{align*}
The additional term is known as the It\^o correction term in quantum stochastic calculus. Morally, the natural-time order is the order induced by the filtration for which one expects $U_t$ to be adapted to. \alert{Expand on the concept of normal order!}

With the unitary evolution $U_t$ one can derive an evolution for any system observable $X_t=U_t^\dagger (X\otimes 1_{\calH_R})U_t$ given by
\[
	\dd X_t = -i \dd B_t^+[D,X_t] + i[X_t,D^\dagger]\dd B_t^- + LX_t\dd t
\]
where
\[
	LX =  2\Re(\gamma_-)D^\dagger X D - \gamma_-D^\dagger DX - \overline{\gamma}_- X D^\dagger D,
\]
is the corresponding Lindblad operator.

\begin{remark}
	The rescaled equation \eqref{eq:rescaled-schrodinger-interaction} may be formally expressed as 
	\[
		U_{t/\lambda^2}^\lambda = I + \sum_{n=1}^\infty (-i)^n\lambda^n\int_0^{t/\lambda^2}\!\!\!\!\cdots \int_0^{t_{n-1}} H_I(t_1)\cdots H_I(t_n)\dd t_1\cdots \dd t_n.
	\]
	In the case when $H_I(t)$ commutes for every $t\ge 0$, the integrand would be a symmetric function of $t_1,\ldots,t_n$, and gives
	\[
		U_{t/\lambda^2}^\lambda = I + \sum_{n=1}^\infty (-i)^n\frac{\lambda^n}{n!}\left(\int_0^{t/\lambda^2}H_I(s)\dd s\right)^n = \exp\left(-i\lambda\int_0^{t/\lambda^2} H_I(s)\dd s\right),
	\]
	i.e., the expectation of $U_{t/\lambda^2}^\lambda$ is the characteristic function of the process
	\[
		W_t^\lambda \coloneqq \lambda\int_0^{t/\lambda^2} H_I(s)\dd s.
	\]
\end{remark}

\subsection{Damped harmonic oscillator}

We consider a simple setup in which a single atom interacts with an electromagnetic field. The total system is given by the composite of the atom and the reservoir systems
\[
	S = \bigl\{ \frF_\mathsf{sym}(\bbC^2),H_S\bigr\},\qquad R = \bigl\{ \frF_\mathsf{sym}(L^2(\bbR^d)),H_R\bigr\},
\]
with the free Hamiltonians
\[
	H_S = \omega_0\,c^+c^-,\qquad H_R = \int_{\bbR^d} \omega(x)\,a_x^+a_x^-\dd x,
\]
where $\{c^+,c^-\}$ is the field operator for $\calH_S=\frF_\mathsf{sym}(\bbC^2)$, $\{a_x^+,a_x^-\}$ are the field operators for $\calH_R=\frF_\mathsf{sym}(L^2(\bbR^d))$, and $\omega$ is a suitable cutoff function.

For the interaction Hamiltonian $H_I$, we consider a dipole approximation of the form
\[
	H_I = \int_{\bbR^d} \chi(x)\bigl[c^-\otimes a_x^+ +c^+\otimes a_x\bigr]\dd x = c^- \otimes A^+ + c^+\otimes A^-,
\]
where $\chi$ are suitable cutoff function and
\[
	A^\pm = \int_{\bbR^d} \chi(x)\,a_x^\pm\dd x.
\]
such that the rescaled total Hamiltonian is given by
\[
	H^\lambda = H_S\otimes 1_{\calH_R} + 1_{\calH_S}\otimes H_R + \lambda H_I.
\]
Define the evolution
\[
	\fru_t^\lambda(a) = e^{it H^\lambda} a e^{-it H^\lambda}.
\]
Then, the Heisenberg equation for $c(t)=\fru_t^\lambda(c\otimes 1_{\calH_R})$ and $a_x^-(t) = \fru_t^\lambda(1_{\calH_S}\otimes a_x)$ reads
\begin{align*}
	\frac{\d}{\d t} c^-(t) &= -i\omega_0\,c^-(t) - i\lambda \int_{\bbR^d} \chi(x)\,a_x^-(t)\dd x \\
	\frac{\d}{\d t} a_x^-(t) &= -i\omega(x)\,a_x^-(t) - \lambda\,\chi(x)\,c(t).
\end{align*}
Solving for $a_x^-(t)$, one obtains
\[
	a_x^-(t) = a_x^- e^{-it\omega(x)} - i\lambda\chi(x)\int_0^t e^{-i(t-s)\omega(x)} c^-(s)\dd s.
\]
Inserting $a_x^-(t)$ into the equation for $c(t)$, we find
\[
	\frac{\d}{\d t} c^-(t) = -i\omega_0 c^-(t) -\int_0^t \gamma(t-s)\,c^-(s)\dd s - i\,\xi_t^-.
\]
where we defined the quantities
\[
	\gamma(r) \coloneqq \lambda^2\int_{\bbR^d} e^{-ir\omega(x)}\chi^2(x)\dd x,\qquad \xi_t^-\coloneqq \lambda \int_{\bbR^d} e^{-it\omega(x)}\chi(x)\,a_x^-\dd x.
\]
Observe that $\xi_t^-$ depends only on the reservoir field operator $a_x^-$, and therefore, acts as an \emph{external force} to the atomic system.

Now let us consider the statistics of $\xi_t^-$ for the Fock vacuum state free state $\Uppsi_\Upomega = \langle \Upomega,\cdot\,\Upomega\rangle$ on $\calH_R$. Since $\Uppsi_\Upomega$ is a Gaussian free state, the statistics of $\xi_t^-$ are uniquely determined by the 2-points correlation functions. Clearly, $\Uppsi_\Upomega$ has zero mean, and therefore,
\[
	\Uppsi_\Upomega(\xi_t^-) = \lambda \int_{\bbR^d} e^{-it\omega(x)}\chi(x)\,\Uppsi_\Upomega(a_x^-)\dd x = 0,\qquad \Uppsi_\Upomega((\xi_t^-)^\dagger) =0.
\]
Moreover, we find
\[
	\Uppsi_\Upomega((\xi_t^-)^\dagger\,\xi_s^-) = \Uppsi_\Upomega(\xi_t^-\,\xi_s^-) = \Uppsi_\Upomega((\xi_t^-)^\dagger\,(\xi_s^-)^\dagger) = 0,
\]
and
\begin{align*}
	\Uppsi_\Upomega(\xi_t^-\,(\xi_s^-)^\dagger) &= \lambda^2\int_{\bbR^d}\int_{\bbR^d} e^{-it(\omega(x)-\omega(y))}\chi(x)\chi(y)\,\Uppsi_\Upomega(a_x^-a^+_y)\dd x\dd y \\
	&= \lambda^2\int_{\bbR^d} e^{-i(t-s)\omega(x)}\chi^2(x)\dd x = \gamma(t-s).
\end{align*}
Hence, $\xi_t^-$ is a mean-zero $\gamma$-correlated \emph{random process} under $\Uppsi$.

To obtain the stochastic limit, we start by rescaling time $t\mapsto t/\lambda^2$ and consider the rescaled quantity $c^{\lambda,\pm}(t)\coloneqq c^\pm(t/\lambda^2)$, which yields
\begin{align}\label{eq:quantum-pre-langevin}
	\dot c^{\lambda,-}(t) = -i\omega_0 c^{\lambda,-}(t) -\int_0^t \gamma^\lambda(t-s)\,c^{\lambda,-}(s)\dd s -i\, \xi_t^{\lambda,-},
\end{align}
with
\[
	\gamma^\lambda(r) \coloneqq \frac{1}{\lambda^2}\langle \chi,e^{-i(r/\lambda^2)\omega}\chi\rangle,\qquad \xi_t^{\lambda,-}\coloneqq \lambda^{-2}\xi_{t/\lambda^2}^-.
\]
From Theorem~\ref{thm:stochastic-limit}, we then establish that
\[
	\gamma^\lambda(r) \longrightarrow \gamma\,\delta(r),\qquad \xi_t^{\lambda,-}\longrightarrow b_t^-\qquad\text{in the sense of correlators,}
\]
with $\gamma = \langle \chi,\delta(\omega)\chi\rangle$.
Consequently, we obtain
\[
	\d c_t^- = -(i\omega_0 + \gamma)\,c_t^-\dd t - i \dd B_t^-,\qquad B_t^- = \int_0^t b_s^-\dd s.
\]


\[
	\dd U_t = -i(c^-\dd B_t^+ U_t + c^- U_t\dd B_t^-) - \gamma_-c^+c^- U_t\dd t
\]
\begin{align*}
	\dd (U_t^\dagger c^- U_t) &= i(c^-\dd B_t^+ U_t + c^+ U_t\dd B_t^-)^\dagger c^-U_t - \overline{\gamma}_- U_t^\dagger c^-c^+c^-U_t \dd t \\
	& -i U_t^\dagger c^-(c^-\dd B_t^+ U_t + c^+ U_t\dd B_t^-) - \gamma_- U_t^\dagger c^- c^+c^- U_t\dd t \\
	&= iU_t^\dagger \dd B_t^- c^+c^-U_t + i\dd B_t^+ U_t^\dagger c^-c^-U_t  - \overline{\gamma}_- U_t^\dagger c^-c^+c^-U_t \dd t
\end{align*} 












