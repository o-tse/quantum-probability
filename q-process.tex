%\section{Quantum measurement processes}
\section{Stochastic Schr\"odinger Equation}

As we have seen, a quantum circuit is essentially a noisy controlled Schr\"odinger equation. In this section, we consider classical noise sources arising from the control mechanism in terms of light-matter interactions. Without going into the details of the physics, we introduce the stochastic Schr\"odinger equation, which captures the evolution of a quantum system driven by classical noise sources.

Let $\calH$ be a Hilbert space and $\scrB(\calH)$ be the space of bounded operators on $\calH$. We also consider the following subspaces of $\scrB(\calH)$ that play a distinguishing role:
\begin{align*}
	\scrU(\calH) &:= \bigl\{ \oba\in \scrB(\calH) : \oba^\dagger \oba = \sfI_\calH \bigr\} &\text{(unitary operators)}\\
	\scrO(\calH) &:= \bigl\{ \oba\in \scrB(\calH) : \oba^\dagger = \oba \bigr\} &\text{(hermitian operators)}
\end{align*}
where $\oba^\dagger \coloneqq \overline\oba^\top$ is the hermitian conjugate of an operator $\oba\in \scrB(\calH)$. 

Given a Hamiltonian $H\in\scrO(\calH)$ and a real-valued smooth path $\alpha_t$, the Schr\"odinger equation driven by $\dot\alpha_t H$ may be expressed as an evolution in the space $\scrU(\calH)$:
\[
	\d U_t = iHU_t\dd \alpha_t,\qquad U_0 = \sfI_\calH.
\]
In particular, the solution may be explicitly expressed as
\[
	U_t = \exp\bigl(i(\alpha_t-\alpha_0) H\bigr)\sfI_\calH.
\]
In the presence of noise, however, $\alpha_t$ is no longer smooth. Yet, if 
\[
	\alpha_t = \alpha_0 + B_t + M_t,
\]



\begin{align*}
	\d U_t = HU_t\,\d t + \sum\nolimits_j V_jU_t\circ \d\alpha_t,
\end{align*}
where $H$, $V_j$ are 




\subsection{Quantum measurements}

Let $\varrho\in\calS(\calH)$ be a state on a Hilbert space $\calH$ and $t\mapsto U_t \in \calU(\calH)$ be a group of unitary actions on $\calH$. Suppose $\sfQ$ is an $\bbR$-valued observable with spectral measure $\pi^\sfQ:\calB(\bbR)\to \calP(\calH)$, i.e., $\pi^\sfQ$ is a projection-valued measure (PVM). Then the probability of measuring an event $A\in\calB(\bbR)$ at time $t^->0$ is given by Born's rule:
\[
	\bbP_\varrho(\lambda_{t^-}^\sfQ\in A) = \tr[\pi^\sfQ(A)\varrho_{t^-}] = \tr[U_t^\dagger\pi^\sfQ(A)U_t \varrho]. 
\]
As measurements generally change the state of the system, L\"uders' rule postulates that, \emph{conditioned} on the event $A\in \calB(\bbR)$, the state ``collapses" to
\[
	\varrho_{t^+} = \scrS(\pi^\sfQ(A),\varrho_{t^-}),\qquad \scrS(\pi,\varrho) \coloneqq \frac{\pi\varrho\,\pi^\dagger}{\tr[\pi\varrho \pi^\dagger]}\in\calS(\calH).
\]
Notice that by definition, $\scrS(\pi,U\varrho\,U^\dagger) = \scrS(\pi U,\varrho)$ and $U\scrS(\pi,\varrho) U^\dagger=\scrS(U\pi,\varrho)$.

The probability of measuring the event $B\in\calB(\bbR)$ at time $t^+>0$ given that the event $A\in\calB(\bbR)$ was measured at time $t>0$ is
\[
	\bbP_\varrho(\lambda_{t^+}^\sfQ\in B\,|\,\lambda_{t^-}^\sfQ\in A) = \tr[\pi^\sfQ(B)\scrS(\pi^\sfQ(A),\varrho_{t^-})] = \frac{\tr[\pi^\sfQ(A\cap B)\varrho_{t^-}]}{\tr[\pi^\sfQ(A)\varrho_{t^-}]}.
\]
In particular, the conditional probability of measuring event $A\in\calB(\bbR)$ at time $t^+>0$ given it was measured at time $t^->0$ is $1$. If $\sfR$ is another $\bbR$-valued observable, then
\[
	\bbP_\varrho(\lambda_{t^+}^\sfR\in B\,|\,\lambda_{t^-}^\sfQ\in A) = \tr[\pi^\sfR(B)\varrho_{t^+}] = \frac{\tr[\pi^\sfR(B)\pi^\sfQ(A)\varrho_{t^-}\pi^\sfQ(A)\pi^\sfR(B)]}{\tr[\pi^\sfQ(A)\varrho_{t^-}]}.
\]
When $\sfR$ and $\sfQ$ commute, then $\pi^\sfR(B)\pi^\sfQ(A) = \pi^\sfQ(A)\pi^\sfR(B)$, and hence,
\[
	\bbP_\varrho(\lambda_{t^+}^\sfR\in B\,|\,\lambda_{t^-}^\sfQ\in A)\bbP_\varrho(\lambda_{t^-}^\sfQ\in A) = \bbP_\varrho(\lambda_{t^+}^\sfQ\in A\,|\,\lambda_{t^-}^\sfR\in B)\bbP_\varrho(\lambda_{t^-}^\sfR\in B),
\]
i.e., Bayes' theorem holds and the usual notion of joint probability is well defined. Clearly, this does not hold when $\sfR$ and $\sfQ$ do not commute. It is this failure of the Bayes theorem that distinguishes quantum probability from classical probability. 

In a similar fashion, one determines the probability of measuring the event $A_2$ at time $t_2^->0$ for $\sfR$ given that the event $A_1$ was measured at time $t_1>0$ for $\sfQ$ from 
\begin{align*}
	\bbP_\varrho(\lambda_{t_2^-}^\sfR\in A_2\,|\,\lambda_{t_1^-}^\sfQ\in A_1) 
	%&= \tr[\pi^\sfR(A_2)\varrho_{t_2^-}],\qquad \varrho_{t_2^-} = U_{t_2-t_1}\scrS(\pi^\sfQ(A_1),\varrho_{t_1^-})U_{t_2-t_1}^\dagger \\
%	&= \tr[\pi^\sfR(A_2)U_{t_2-t_1}\varrho_{t_1^+}U_{t_2-t_1}^\dagger] \\
	&= \tr[\pi^\sfR(A_2)U_{t_2-t_1}\scrS(\pi^\sfQ(A_1),\varrho_{t_1^-})U_{t_2-t_1}^\dagger] \\
%	&= \frac{\tr[\pi^\sfR(A_2)U_{t_2-t_1}\pi^\sfQ(A_1)U_{t_1}\varrho\, U_{t_1}^\dagger\pi^\sfQ(A_1)U_{t_2-t_1}^\dagger]}{\tr[\pi^\sfQ(A_1)U_{t_1}\varrho U_{t_1}^\dagger]} \\
	&= \tr[\pi^\sfR(A_2)U_{t_2-t_1} \scrS(\pi^\sfQ(A_1)U_{t_1},\varrho\, )U_{t_2-t_1}^\dagger] \\
	&= \tr[\pi^\sfR(A_2)\scrS(U_{t_2-t_1}\pi^\sfQ(A_1) U_{t_1}, \varrho)] = \tr[\pi^\sfR(A_2)\varrho_{t_2^-}].
\end{align*}
As before, upon measuring the event $A_2$ at time $t_2>0$, L\"uders' rule implies
\[
	\varrho_{t_2^+} = \scrS(\pi^\sfR(A_2),\varrho_{t_2^-}) = \scrS(\pi^\sfR(A_2)U_{t_2-t_1}\pi^\sfQ(A_1) U_{t_1},\varrho).
\]
Restricting ourselves to a single observable $\sfQ$, 
